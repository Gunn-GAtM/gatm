\documentclass[../key.tex]{subfiles}

\begin{document}

\section{From Snaps to Flips}

\begin{outer_problem}[start=1]
\item Is the list of six operations complete? (Are there any other isometries of the equilateral triangle that preserve its shape and location?)
\end{outer_problem}

\noindent There are no other isometries for this triangle; our list of operations is complete. To see why, note that the vertices must exchange places. At most there is $3!=6$ ways to do this, so we have already achieved the maximum possible number of isometries.

\begin{figure}[h]
	\begin{center}
		\begin{minipage}[b]{\textwidth}
			\centering
			\begin{tabular}{c|cccccc}
				\hline
				$\cdot$ & $I$ & $A$ & $B$ & $C$ & $D$ & $E$ \\ \hline
				\rowcolor{light-gray}
				$I$    &   &   &   &   &   &   \\ 
				$A$    &   &   &   &   & $B$  &   \\ 
				\rowcolor{light-gray}
				$B$    &   &   &   &   &   &   \\ 
				$C$    &   &   &   &   &   &   \\
				\rowcolor{light-gray} 
				$D$    &   &   &   &   &   &   \\ 
				$E$    &   &   &   &   &   &   \\ \hline
			\end{tabular}
			\vspace*{0.5\baselineskip}
		\end{minipage}
	\end{center}
	\vspace*{-2\baselineskip}
	\begin{center}
		\begin{minipage}[t]{\textwidth}
			\caption{Unfilled $D_3$ group table.}
			\label{fig:sbstable}
		\end{minipage}
	\end{center}
	\vspace*{-2\baselineskip}
\end{figure}

\begin{outer_problem}
\item As with the snap group, we can make a group table for the dihedral group. Fill out a table like the one in Figure~\ref{fig:sbstable} in your notebook. Like the snap group table, the top row indicates what operation is done first and the left column indicates what's done second. In other words, $XY$ is in the $X$'s' row and $Y$'s column. $AD=B$ is done for you.
\end{outer_problem}

\noindent The completed table is shown in Figure~\ref{fig:complete_sbs_table}.

\begin{figure}[h]
	\begin{center}
		\begin{minipage}[b]{0.45\textwidth}
			\centering
			\begin{tabular}{c|cccccc}
				\hline
				$\cdot$ & $I$ & $A$ & $B$ & $C$ & $D$ & $E$ \\ \hline
				\rowcolor{light-gray}
				$I$ & $I$ & $A$ & $B$ & $C$ & $D$ & $E$ \\
				$A$ & $A$ & $I$ & $D$ & $E$ & $B$ & $C$ \\
				\rowcolor{light-gray}
				$B$ & $B$ & $E$ & $I$ & $D$ & $C$ & $A$ \\
				$C$ & $C$ & $D$ & $E$ & $I$ & $A$ & $B$ \\
				\rowcolor{light-gray}
				$D$ & $D$ & $C$ & $A$ & $B$ & $E$ & $I$ \\
				$E$ & $E$ & $B$ & $C$ & $A$ & $I$ & $D$ \\ \hline
			\end{tabular}
			\vspace*{0.5\baselineskip}
		\end{minipage}
		\hfill
		\begin{minipage}[b]{0.45\textwidth}
			\centering
			\begin{tabular}{c|cccccc}
				\hline
				$\bullet$ & $I$ & $A$ & $B$ & $C$ & $D$ & $E$ \\ \hline
				\rowcolor{light-gray}
				$I$ & $I$ & $A$ & $B$ & $C$ & $D$ & $E$ \\
				$A$ & $A$ & $I$ & $E$ & $D$ & $C$ & $B$ \\
				\rowcolor{light-gray}
				$B$ & $B$ & $D$ & $I$ & $E$ & $A$ & $C$ \\
				$C$ & $C$ & $E$ & $D$ & $I$ & $B$ & $A$ \\
				\rowcolor{light-gray}
				$D$ & $D$ & $B$ & $C$ & $A$ & $E$ & $I$ \\
				$E$ & $E$ & $C$ & $A$ & $B$ & $I$ & $D$ \\ \hline
			\end{tabular}
			\vspace*{0.5\baselineskip}
		\end{minipage}
	\end{center}
	\vspace*{-2\baselineskip}
	\begin{center}
		\begin{minipage}[t]{0.45\textwidth}
			\caption{Completed $D_3$ group table.}
			\label{fig:complete_sbs_table}
		\end{minipage}
		\hfill
		\begin{minipage}[t]{0.45\textwidth}
			\caption{Completed $S_3$ group table from the last chapter.}
			\label{fig:complete_sts_table}
		\end{minipage}
	\end{center}
	\vspace*{-2\baselineskip}
\end{figure}

\begin{outer_problem}
\item What is the relationship between the tables for the snap group $S_3$ and the dihedral group $D_3$?
\end{outer_problem}

% amazing usage of the n dash here! :)
\noindent$D_3$'s table is $S_3$'s table flipped over the top-left--bottom-right diagonal, and vice versa. Contrast $D_3$ from Figure~\ref{fig:complete_sbs_table} to $S_3$ in Figure~\ref{fig:complete_sts_table}. If these were matrices, one would be the transpose of the other: we'll get to that later.

\begin{outer_problem}
\item Check your understanding by defining isomorphic in your own words.
\end{outer_problem}

\noindent (Answers may vary.)

Isomorphic means that two groups have the same structure.
Isomorphic means that there is a correspondence between the elements of two groups so that the correspondence preserves the order.
In the language of abstract algebra, an isomorphism between groups $A$ and $B$ exists if there is a homomorphism from $A$ to $B$ and from $B$ to $A$.

\begin{figure}[h]
	\begin{center}
		\begin{minipage}[b]{\textwidth}
			\centering
			\begin{tabular}{c|ccc}
				\hline
				$\cdot$ & $I$ & $D$ & $E$ \\ \hline
				\rowcolor{light-gray}
				$I$ & $I$ & $D$ & $E$ \\
				$D$ & $D$ & $E$ & $I$ \\
				\rowcolor{light-gray}
				$E$ & $E$ & $I$ & $D$ \\ \hline
			\end{tabular}
			\vspace*{0.5\baselineskip}
		\end{minipage}
	\end{center}
	\vspace*{-2\baselineskip}
\end{figure}

\begin{outer_problem}
\item
\begin{inner_problem}[start=1,leftmargin=25pt]
	\item Make a table for only the rotations of $D_3$, a subgroup of $D_3$.
\end{inner_problem}
\end{outer_problem}

\noindent The table is shown above. Note that the identity element $I$ is a rotation of $0$.

Interestingly, this subgroup is a commutative group, also known as an abelian group.

\begin{inner_problem}
\item Which subgroup of the snap group $S_3$ is isomorphic to the subgroup in (a)?
\end{inner_problem}

\noindent The same elements (nominally) make the same subgroup:

\begin{figure}[h]
	\begin{center}
		\begin{minipage}[b]{\textwidth}
			\centering
			\begin{tabular}{c|ccc}
				\hline
				$\cdot$ & $I$ & $D$ & $E$ \\ \hline
				\rowcolor{light-gray}
				$I$ & $I$ & $D$ & $E$ \\
				$D$ & $D$ & $E$ & $I$ \\
				\rowcolor{light-gray}
				$E$ & $E$ & $I$ & $D$ \\ \hline
			\end{tabular}
			\vspace*{0.5\baselineskip}
		\end{minipage}
	\end{center}
	\vspace*{-2\baselineskip}
\end{figure}

\begin{outer_problem}
\item What shape's dihedral group is isomorphic to
\end{outer_problem}

\begin{inner_problem}[start=1]
\item the two post snap group $S_2$?
\end{inner_problem}

\noindent The dihedral group of a line segment is isomorphic to $S_2$. After all, you can only reflect it over its midpoint, which is the other element of $S_2$ besides the identity. We can also think of this as permuting the two endpoints or vertices of a line segment.

\begin{inner_problem}
\item the one post snap group $S_1$?
\end{inner_problem}

\noindent The dihedral group of a point is isomorphic to $S_1$, because the only element is the identity element. This is permuting the one vertex of a point.

\begin{inner_problem}
\item the four post snap group $S_4$?
\end{inner_problem}

\noindent For this question we need to think $3$ dimensions. There are four vertices to permute, but we can't do that on a square since diagonal points will remain on diagonals, as shown in Figure~\ref{fig:impossiblus_square}.

\begin{figure}[h]
	\begin{center}
		\begin{minipage}[b]{\textwidth}
			\centering
			\begin{asy}[width=0.5\textwidth]
			draw((0,0)--(1,0)--(1,1)--(0,1)--cycle);
			
			label("$A$", (0,0), SW);
			label("$B$", (1,0), SE);
			label("$C$", (1,1), NE);
			label("$D$", (0,1), NW);
			
			draw(shift(2.5,0)*((0,0)--(1,0)--(1,1)--(0,1)--cycle));
			
			label("$A'$", (2.5,0), SW);
			label("$C'$", (3.5,0), SE);
			label("$B'$", (3.5,1), NE);
			label("$D'$", (2.5,1), NW);
			
			label("$\not\longrightarrow$", (3.5/2, 0.5));
			\end{asy}
		\end{minipage}
	\end{center}
	\vspace*{-2\baselineskip}
	\begin{center}
		\begin{minipage}[t]{\textwidth}
			\caption{At right is a valid permutation of the vertices, but not a valid isometry of the square.}
			\label{fig:impossiblus_square}
		\end{minipage}
	\end{center}
	\vspace*{-2\baselineskip}
\end{figure}

\noindent Instead, we choose the regular tetrahedron, so that there are no ``diagonals''; every permutation is achievable. Note that rotations and reflections are now in $3$ dimensional space, which is a bit difficult to visualize. A sample rotation is depicted in Figure~\ref{fig:tetra_element}.

\begin{asydef}
import three;

triple[] generate_tetrahedron() {
	triple[] ret = {};

	triple A = (0,0,0);
	triple B = (1,0,0); // 1 unit in the x direction
	triple C = (1/2, sqrt(3)/2, 0); // mid way between A--B, then jutting out sqrt(3)/2 in y

	// the height of this tetrahedron is sqrt(2/3)
	// the peak vertex is above the centroid of the base
	triple D = (A+B+C)/3 + (0,0,sqrt(2/3));

	ret[0] = A;
	ret[1] = B;
	ret[2] = C;
	ret[3] = D;

	return ret;
}

void draw_tetra(triple[] vertices) {
	for (int i = 0; i < 4; ++i) {
		for (int j = i + 1; j < 4; ++j) {
			draw(vertices[i]--vertices[j], i == 0 ? dashed : currentpen);
		}
	}
}

\end{asydef}

\begin{figure}[h]
	\begin{center}
		\begin{minipage}[b]{\textwidth}
			\centering
			\begin{asy}[width=0.5\textwidth]
			currentprojection = orthographic(50,40,20);
			
			triple[] tetra = generate_tetrahedron();
			
			draw_tetra(tetra);
			
			label("$A$", tetra[0], S);
			label("$B$", tetra[1], SW);
			label("$C$", tetra[2], SE);
			label("$D$", tetra[3], N);
			
			triple[] new_tetra = generate_tetrahedron();
			
			for (int i = 0; i < 4; ++i) new_tetra[i] = 3 * (0, sqrt(3)/2, 0) + new_tetra[i];
			
			label("$D'$", new_tetra[0], 1.5*S);
			label("$C'$", new_tetra[1], SW);
			label("$B'$", new_tetra[2], SE);
			label("$A'$", new_tetra[3], N);
			
			draw_tetra(new_tetra);
			
			triple axis1 = (tetra[0]+tetra[3])/2;
			triple axis2 = (tetra[1]+tetra[2])/2;
			
			triple rot_mc = 2*axis2 - axis1;
			
			draw((2*axis1 - axis2) -- rot_mc, dotted);
			
			dot(axis1);
			dot(axis2);
			
			label("$l$", rot_mc, NW);
			
			triple unitify(triple d) {
				return d * 1 / sqrt(d.x*d.x+d.y*d.y+d.z*d.z);
			}
			
			triple rot_center = 1.5*axis2 - 0.5*axis1;
			
			triple rcsdv = 0.4 * (0.5*axis2 - 0.5*axis1); // rotation center deviant
			
			triple rot_s = rotate(-90, tetra[2]-tetra[1]) * rcsdv + rot_center;
			triple rot_e = rotate(90, tetra[2]-tetra[1]) * rcsdv + rot_center;
			triple rot_i = rotate(90, axis1, axis2) * rot_e;
			
			draw(rot_s..rot_i..rot_e, Arrow3);
			
			
			triple center1 = (tetra[0] + tetra[1] + tetra[2] + tetra[3]) / 4;
			triple center2 = (new_tetra[0] + new_tetra[1] + new_tetra[2] + new_tetra[3]) / 4;
			
			path3 center_path = center1--center2;
			
			draw(point(center_path,0.3)--point(center_path,0.7),Arrow3);
			
			label(YZ()*"Rotate $180^\circ$ about axis $l$", center_path, N);
			\end{asy}
		\end{minipage}
	\end{center}
	\vspace*{-2\baselineskip}
	\begin{center}
		\begin{minipage}[t]{\textwidth}
			\caption{A rotation of the tetrahedron (orthographic view).}
			\label{fig:tetra_element}
		\end{minipage}
	\end{center}
	\vspace*{-2\baselineskip}
\end{figure}

\begin{inner_problem}
\item the five post snap group $S_5$?
\end{inner_problem}

\noindent This is isomorphic to the dihedral group of the $4$-dimensional equivalent of the tetrahedron, also known as the regular $4$-simplex. A projection is shown in Figure~\ref{fig:four_simplex_attempt}, but it cannot be faithfully represented on this paper.

\begin{figure}[h]
	\begin{center}
		\begin{minipage}[b]{\textwidth}
			\centering
			\begin{asy}[width=0.3\textwidth]
			
			triple[] vertices = generate_tetrahedron();
			
			currentprojection = perspective(5,5.6,2);
			
			draw_tetra(vertices);
			
			
			label("$A$", vertices[0], NE);
			label("$B$", vertices[1], SW);
			label("$C$", vertices[2], E);
			label("$D$", vertices[3], N);
			
			triple E = (vertices[0] + vertices[1] + vertices[2] + vertices[3])/4;
			
			for (int i = 0; i < 4; ++i) {
				draw(E--vertices[i],dashed);
				dot(vertices[i]);
			}
			
			label("$E$", E, S);
			
			dot(E);
			
			\end{asy}
		\end{minipage}
	\end{center}
	\vspace*{-2\baselineskip}
	\begin{center}
		\begin{minipage}[t]{\textwidth}
			\caption{A 3D projection of the regular $4$-simplex. In a true realization, every line segment here would be the same length.}
			\label{fig:four_simplex_attempt}
		\end{minipage}
	\end{center}
	\vspace*{-2\baselineskip}
\end{figure}

\begin{outer_problem}
\item Find a combination of $A$ and $D$ that yields $C$.
\end{outer_problem}

% heyyy why is \degree banned in answers...?
\begin{outer_problem}
\item We call $A$ and $D$ \textbf{generators} of the group because every element of the group is expressible as some combination of $A$s and $D$s. For convenience, let's call $A$ ``$f$'' since it's a flip, and call $D$ ``$r$'' meaning a $120\deg$ rotation counterclockwise. Then, for example, $fr^2$ is a rotation of $2\cdot 120\deg = 240\deg$, followed by a flip across the $A$ axis, equivalent to our original $C$ (see Figure~\ref{fig:fr2}). Make a new table using $I$, $r$, $r^2$, $f$, $fr$, and $fr^2$ as elements, like the one in Figure~\ref{fig:alttable}. \textit{Note that the element order is different!}
\end{outer_problem}

\begin{figure}[h]
	\begin{center}
		\begin{minipage}[b]{\textwidth}
			\centering
			\begin{tabular}{c|cccccc}
				\hline
				$\cdot$ & $I$ & $r$ & $r^2$ & $f$ & $fr$ & $fr^2$ \\ \hline 
				\rowcolor{light-gray}
				$I$    &   &   &   &   &   &   \\ 
				$r$    &   &   &   & $fr^2$  &   &   \\  
				\rowcolor{light-gray}
				$r^2$    &   &   &   &   &   &   \\ 
				$f$    &   &   &   &   &   &   \\  
				\rowcolor{light-gray}
				$fr$    &   &   &   &   &   &   \\ 
				$fr^2$    &   &   &   &   &   &   \\ \hline
			\end{tabular}
			\vspace*{0.5\baselineskip}
		\end{minipage}
	\end{center}
	\vspace*{-2\baselineskip}
	\begin{center}
		\begin{minipage}[t]{\textwidth}
			\captionof{figure}{Unfilled alternate $D_3$ table.}
			\label{fig:alttable}
		\end{minipage}
	\end{center}
	\vspace*{-2\baselineskip}
\end{figure}

\noindent The filled table is shown in Figure~\ref{fig:filled_alttable} below.

\begin{figure}[h]
	\begin{center}
		\begin{minipage}[b]{\textwidth}
			\centering
			\begin{tabular}{c|cccccc}
				\hline
				$\cdot$ & $I$ & $r$ & $r^2$ & $f$ & $fr$ & $fr^2$ \\ \hline 
				\rowcolor{light-gray}
				$I$ & $I$ & $r$ & $r^2$ & $f$ & $fr$ & $fr^2$ \\
				$r$ & $r$ & $r^2$ & $I$ & $fr^2$ & $f$ & $fr$ \\ 
				\rowcolor{light-gray}
				$r^2$ & $r^2$ & $I$ & $r$ & $fr$ & $fr^2$ & $f$ \\
				$f$ & $f$ & $fr$ & $fr^2$ & $I$ & $r$ & $r^2$ \\ 
				\rowcolor{light-gray}
				$fr$ & $fr$ & $fr^2$ & $f$ & $r^2$ & $I$ & $r$ \\
				$fr^2$ & $fr^2$ & $f$ & $fr$ & $r$ & $r^2$ & $I$ \\ \hline
			\end{tabular}
			\vspace*{0.5\baselineskip}
		\end{minipage}
	\end{center}
	\vspace*{-2\baselineskip}
	\begin{center}
		\begin{minipage}[t]{\textwidth}
			\captionof{figure}{Completed alternate $D_3$ table.}
			\label{fig:filled_alttable}
		\end{minipage}
	\end{center}
	\vspace*{-2\baselineskip}
\end{figure}

\noindent Note that $I=I$, $A=f$, $B=fr$, $C=fr^2$, $D=r$, and $E=r^2$.

\begin{outer_problem}
\item What other pairs of elements could you have used to generate the table?
\end{outer_problem}

\noindent You could also use any of the following pairs: $\{A,E\}$, $\{B,D\}$, $\{B,E\}$, $\{C,D\}$, $\{C,E\}$, $\{A,B\}$, $\{B,C\}$, $\{A,C\}$. In essence, you can generate it with any rotation element and any reflection element, or with any two reflection elements.

\begin{outer_problem}
\item Notice the $3\times 3$ table of a group you've already described in the top-left corner of your table. What is it, and what are the two possible generators of this three-element group?
\end{outer_problem}

\noindent This is the cyclic group of order $3$, $C_3$, also known as the rotation group of the equilateral triangle. The two possible generators are $r$ and $r^2$.

\begin{outer_problem}
\item Explain why each element of the dihedral group $D_3$ has the period it has.
\end{outer_problem}

\noindent$I$ has a period of $1$ because it is the identity. $A,B,C$ have periods of $2$ because they are reflections, so they are their own inverse transformation. $D$ and $E$ are rotations of a multiple of $1/3$ of a turn. Since $3$ is a prime, they take $3$ iterations to resolve, and thus have period $3$.

\begin{outer_problem}
\item Some pairs of elements of the dihedral group are two-element subgroups. Which pairs are they?
\end{outer_problem}

\noindent These would be the pairs ${I,A}$, ${I,B}$, and ${I,C}$, since $A\cdot A=B\cdot B=C\cdot C = I$ so the subgroup is closed. These are shown in Figure~\ref{fig:two_elem_subgroups}.

\begin{figure}[h]
	\begin{center}
		\begin{minipage}[b]{0.3\textwidth}
			\centering
			\begin{tabular}{c|cc}
				\hline
				$\cdot$ & $I$ & $A$ \\ \hline
				\rowcolor{light-gray}
				$I$ & $I$ & $A$ \\
				$A$ & $A$ & $I$ \\ \hline
			\end{tabular}
			\vspace*{0.5\baselineskip}
		\end{minipage}
		\hfill
		\begin{minipage}[b]{0.3\textwidth}
			\centering
			\begin{tabular}{c|cc}
				\hline
				$\cdot$ & $I$ & $B$ \\ \hline
				\rowcolor{light-gray}
				$I$ & $I$ & $B$ \\
				$B$ & $B$ & $I$ \\ \hline
			\end{tabular}
			\vspace*{0.5\baselineskip}
		\end{minipage}
		\hfill
		\begin{minipage}[b]{0.3\textwidth}
			\centering
			\begin{tabular}{c|cc}
				\hline
				$\cdot$ & $I$ & $C$ \\ \hline
				\rowcolor{light-gray}
				$I$ & $I$ & $C$ \\
				$C$ & $C$ & $I$ \\ \hline
			\end{tabular}
			\vspace*{0.5\baselineskip}
		\end{minipage}
	\end{center}
	\vspace*{-2\baselineskip}
	\begin{center}
		\begin{minipage}[t]{\textwidth}
			\captionof{figure}{The three two-element subgroups.}
			\label{fig:two_elem_subgroups}
		\end{minipage}
	\end{center}
	\vspace*{-2\baselineskip}
\end{figure}

\begin{outer_problem}
\item One of the elements forms a one-element subgroup. Which is it?
\end{outer_problem}

\noindent The element $I$ forms the so-called trivial group, or the only group of order $1$; this is shown in Figure~\ref{fig:trivial_group}. It is not very interesting.

\begin{figure}[h]
	\begin{center}
		\begin{minipage}[b]{\textwidth}
			\centering
			\begin{tabular}{c|c}
				\hline
				$\cdot$ & $I$ \\ \hline
				\rowcolor{light-gray}
				$I$ & $I$ \\ \hline
			\end{tabular}
			\vspace*{0.5\baselineskip}
		\end{minipage}
	\end{center}
	\vspace*{-2\baselineskip}
	\begin{center}
		\begin{minipage}[t]{\textwidth}
			\captionof{figure}{The trivial group.}
			\label{fig:trivial_group}
		\end{minipage}
	\end{center}
	\vspace*{-2\baselineskip}
\end{figure}

\begin{outer_problem}
\item The addition of two numbers is a binary operation, while the addition of three numbers is not. In logic, $\land$ (and) and $\lor$ (or) are binary operations, but $\lnot$ (not) is not. Define binary operation in your own words, and name some other binary operations.
\end{outer_problem}

\noindent (Answers may vary.)

A binary operation is an operation with two arguments.

Some binary operations:
\begin{enumerate}
\begin{multicols}{4}
\item multiplication
\item exponentiation
\item addition
\item subtraction
\item division
\item modulo operator
\item bitwise OR
\item bitwise AND
\item snap operation ($\bullet$)
\item function convolution
\end{multicols}
\end{enumerate}

\begin{outer_problem}
\item In your original dihedral group table, what is
\end{outer_problem}

\begin{inner_problem}[start=1]
\item the identity element?
\end{inner_problem}

\noindent The identity element is $I$.

\begin{inner_problem}
\item the inverse of $A$?
\end{inner_problem}

\noindent The inverse of $A$ is also $A$, since it is a reflection.

\begin{inner_problem}
\item the inverse of $E$?
\end{inner_problem}

\noindent The inverse of $E$ is $D$, since $240^\circ+120^\circ\equiv 0^\circ$.

\end{document}
