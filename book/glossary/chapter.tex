\documentclass[../textbook.tex]{subfiles}

\begin{document}

\section{Glossary}
\setcounter{problem_i}{0}

\begin{description}[align=left]

\item[bijection] one-to-one correspondence. If a function $f$ is bijective, it has an inverse $f^{-1}$; more formally, a function is bijective if it is both injective and surjective.

\item[binary operation] an operation acting on two elements.

\item[cardinality] size of a group; number of elements. Also known as order.

\item[closure] the property that operation on elements of a group always produces an element of that same group.

\item[collinear] on the same line.

\item[complex conjugate] the complex conjugate of the complex number $z=a+bi$ has the same real part as $z$ but a negated imaginary part; $\overline{z}=a-bi$.

\item[DeMoivre's theorem] $$(r_1 (\cos \theta + i \sin \theta)) (r_2 (\cos \phi + i \sin \phi)) = (r_1r_2) (\cos(\theta + \phi) + i \sin(\theta + \phi)).$$ In other words, when multiplying two complex numbers, we add their angles and multiply their magnitudes.

\item[dihedral group] the group of rotational and reflective symmetries of a regular polygon.

\item[eigenvalue] the scalar multiple that is associated with the eigenvector.

\item[eigenvector] a vector which when operated on by a given matrix gives a scalar multiple of itself.

\item[Euler's totient function] a function $\phi(n)$ that tells how many numbers are relatively prime to $n$. Formally defined mathematically as: for prime factorization of an integer $n={p_1}^{n_1}{p_2}^{n_2}{p_3}^{n_3}\cdot\cdot\cdot{p_k}^{n_k}$, $\phi(n)$ (symbol for Euler's totient function) is equal to $\phi(n)=n(1-\frac{1}{p_1})(1-\frac{1}{p_2})\cdot\cdot\cdot(1-\frac{1}{p_k})$.

\item[generating set] a set of elements which can generate a group by repeatedly applying the group operator to these elements.

\item[generator] element that can generate the entire group by a series of operations.

\item[group] a set of elements, finite or infinite, formed by a certain binary operation that satisfies the four fundamental properties: closure, associativity, identity, and invertibility.

\item[identity element] an element $I$, when acted on another element $A$ via a group's binary operation $\cdot$, gives $A$; $I\cdot A = A$ for all $A$.

\item[image] output of a transformation given a preimage.

\item[injective function] a function that maps distinct elements of its domain to distinct elements of its codomain.

\item[isometry] a linear transformation preserving length.

\item[isomorphism] a bijection, or one-to-one correspondence, between two groups which preserves the group's structure.

\item[linear mapping] a mapping in which all lines are mapped to lines and the origin remains fixed.

\item[linearly independent] (of an eigenvector) two vectors that are not multiples of each other; having different directions.

\item[matrix decomposition] decomposing a transformation matrix into simpler transformations.

\item[order] size of a group; number of elements. Also known as cardinality.

\item[period] number of times an element of a group has to be operated on itself to yield an identity element of that group.

\item[permutation] an order of things in which they can be arranged.

\item[preimage] input of a transformation.

\item[shear] a linear transformation where all points along a particular line remain fixed, while other points are shifted parallel to the fixed line by a distance proportional to their perpendicular distance from the fixed line.

\item[surjective function] a function that has every values of its codomain pointed at by at least one element in the domain.

\item[transportation matrix] a square matrix connecting vertices of a graph, sometimes known as an adjacency matrix.

\item[unit vector] a vector with a magnitude of $1$.

\end{description}

\end{document}
