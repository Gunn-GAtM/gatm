\documentclass[../key.tex]{subfiles}

\begin{document}

\section{Mapping the Plane with Matrices}

\begin{center}
\centering
\begin{minipage}{0.3\textwidth}
$$\mathop{\left[ \begin{array}{cc} 2 & 3 \\ -1 & 1 \end{array}\right]}^{M}
\mathop{\left[ \begin{array}{c} 2 \\ 1 \end{array}\right]}^{P} = \mathop{\left[ \begin{array}{c} 7 \\ -1 \end{array} \right]}^{P'}$$
\end{minipage}\hfill
\begin{minipage}{0.5\textwidth}
\begin{asy}[width=\textwidth]
pair A = (2, 1);
pair B = (7, -1);

draw((-2,0)--(8,0), Arrow);
draw((0,-2)--(0,3), Arrow);

dot(A);
dot(B);

label("$P=\left[ \begin{array}{c} 2 \\ 1 \end{array}\right]$", A, NE);
label("$P'=\left[ \begin{array}{c} 7 \\ -1 \end{array} \right]$", B, SE);

path between = A--B;
draw(point(between, 0.04) -- point(between, 0.96), Arrow);
\end{asy}
\end{minipage}
\begin{minipage}{0.3\textwidth}
\captionof{figure}{Matrix multiplication is a transformation.}
\label{fig:random_matrix}
\end{minipage}
\end{center}

\begin{outer_problem}[start=1]
\item
\end{outer_problem}

\begin{inner_problem}[start=1]
\item Use the $2\times 2$ matrix from Figure~\ref{fig:random_matrix} to operate on the points $(0,0)$, $(1,0)$, and $(0,1)$. What are their images? Graph them.
\end{inner_problem}

The matrix is $M=\begin{bmatrix} 2 & 3 \\ -1 & 1 \end{bmatrix}$. We multiply this by the column vectors $\begin{bmatrix} 0 \\ 0 \end{bmatrix}$, $\begin{bmatrix} 1 \\ 0 \end{bmatrix}$ and $\begin{bmatrix} 0 \\ 1 \end{bmatrix}$ to get

\begin{align*}
\begin{bmatrix} 2 & 3 \\ -1 & 1 \end{bmatrix}\begin{bmatrix} 0 \\ 0 \end{bmatrix} &= \begin{bmatrix} 0 \\ 0 \end{bmatrix} \\
\begin{bmatrix} 2 & 3 \\ -1 & 1 \end{bmatrix}\begin{bmatrix} 1 \\ 0 \end{bmatrix} &= \begin{bmatrix} 2 \\ -1 \end{bmatrix} \\
\begin{bmatrix} 2 & 3 \\ -1 & 1 \end{bmatrix}\begin{bmatrix} 0 \\ 1 \end{bmatrix} &= \begin{bmatrix} 3 \\ 1 \end{bmatrix}. \\
\end{align*}

Thus, the right hand sides are the images of the left hand side. I graphed the transformation in Figure~\ref{fig:random_mtrx_transformation}.

\begin{center}
\begin{asy}[width=0.5\textwidth]
import graph;

pair A = (0,0);
pair B = (1,0);
pair C = (0,1);

pair Ap = (0,0);
pair Bp = (2,-1);
pair Cp = (3,1);

dot(A);
dot(B);
dot(C);
dot(Ap);
dot(Bp);
dot(Cp);

label("$(0,0)=A'=A$", A, SW);
label("$B=(1,0)$", B, N);
label("$C=(0,1)$", C, NE);

label("$B'=(2,-1)$", Bp, S);
label("$C'=(3,1)$", Cp, N);

draw(point(B--Bp,0.1)--point(B--Bp,0.9), Arrow);
draw(point(C--Cp,0.1)--point(C--Cp,0.9), Arrow);

xaxis("$x$");
yaxis("$y$");

\end{asy}
\captionof{figure}{The matrix operating on points $(0,0)$, $(1,0)$, and $(0,1)$.}
\label{fig:random_mtrx_transformation}
\end{center}

\begin{inner_problem}
\item The preimage includes two perpendicular \textbf{unit vectors}, $(0,1)$ and $(1,0)$. What is the (i) ratio of the lengths of their images and (ii) angle between the images?
\end{inner_problem}

The (i) ratio of the lengths of their images is

$$\frac{|C'|}{|B'|} = \frac{\sqrt{3^2+1^2}}{\sqrt{2^2+(-1)^2}} = \sqrt{\frac{10}{5}} = \sqrt{2}.$$

The (ii) angle between their images is a bit less straightforward, but we can compute it by summing the angle from $C'$ to the $x$-axis with the (positive) angle from $B'$ to the $x$-axis:

$$\tan^{-1} \frac{1}{3} + \left|\tan^{-1} \frac{-1}{2}\right| = \frac{\pi}{4} = 45^\circ.$$

Oh. Well another way to find the angle is to draw the $45-45-90$ triangle between $A$, $B'$ and $C'$, which shows that $\angle C'AB' = 45^\circ$. We know it's $45-45-90$ because the side lengths are $\sqrt{5}$, $\sqrt{5}$ and $\sqrt{10}$ as determined by the Pythagorean Theorem. This triangle is shown in Figure~\ref{fig:succulent_triangle}.

\begin{center}
\begin{asy}[width=0.5\textwidth]
import graph;

pair Ap = (0,0);
pair Bp = (2,-1);
pair Cp = (3,1);

dot(Ap);
dot(Bp);
dot(Cp);

label("$(0,0)=A'=A$", Ap, SW);

label("$B'=(2,-1)$", Bp, S);
label("$C'=(3,1)$", Cp, N);

draw(Ap--Bp--Cp--cycle);
label("$\sqrt{5}$", Ap--Bp, SW);
label("$\sqrt{5}$", Bp--Cp, SE);
label("$\sqrt{10}$", Cp--Ap, NW);

xaxis("$x$");
yaxis("$y$");

\end{asy}
\captionof{figure}{A helpful $45-45-90$ triangle $\triangle AB'C'$.}
\label{fig:succulent_triangle}
\end{center}

\begin{inner_problem}
\item You can conclude that multiplication by matrices does not, in general, preserve which two quantities between the image and preimage?
\end{inner_problem}

It does not preserve the ratio of lengths or the angle between two vectors. After all, the angle was $90^\circ$ and is now $45^\circ$. Also, the ratio used to be $1$, but is now $\sqrt{2}$ (or $\frac{\sqrt{2}}{2}$).

\begin{outer_problem}
\item \label{prob:consolidate_matrix}
\end{outer_problem}

\begin{inner_problem}[start=1]
\item Now, use the $2\times 2$ matrix from Figure~\ref{fig:random_matrix} to operate on each of these points: $(2,1)$, $(1,0)$, $(0,-1)$ and $(-1,-2)$. Do this by consolidating all the points into one matrix, with each point as a column vector, then performing a multiplication:
$$\left[\begin{array}{cc}2 & 3 \\ -1 & 1 \end{array}\right]
\left[\begin{array}{cccc}2 & 1 & 0 & -1 \\ 1 & 0 & -1 & -2\end{array}\right]
=\left[\begin{array}{cccc}\phantom{0} & \phantom{0} & \phantom{0} & \phantom{0} \\ \phantom{0}\end{array}\right].$$
\end{inner_problem}

We perform the multiplication:

$$\begin{bmatrix} 2 & 3 \\ -1 & 1 \end{bmatrix}
\begin{bmatrix} 2 & 1 & 0 & -1 \\ 1 & 0 & -1 & -2 \end{bmatrix}
= \begin{bmatrix}
7 & 2 & -3 & -8 \\
-1 & -1 & -1 & -1
\end{bmatrix}.$$

\begin{inner_problem}
\item Graph and label the preimage and the image of each point onto the same set of axes.
\end{inner_problem}

The graph is shown in Figure~\ref{fig:preimage_and_image_x}.

\begin{center}
\begin{asy}[width=0.6\textwidth]
import graph;

pair A = (2,1);
pair B = (1,0);
pair C = (0,-1);
pair D = (-1,-2);

pair Ap = (7,-1);
pair Bp = (2,-1);
pair Cp = (-3,-1);
pair Dp = (-8,-1);

dot(A);
dot(B);
dot(C);
dot(D);
dot(Ap);
dot(Bp);
dot(Cp);
dot(Dp);

label("$A$", A, N);
label("$B$", B, N);
label("$C$", C, NE);
label("$D$", D, N);

label("$A'$", Ap, S);
label("$B'$", Bp, S);
label("$C'$", Cp, S);
label("$D'$", Dp, S);

xaxis("$x$");
yaxis("$y$");
\end{asy}
\captionof{figure}{The preimage and image.}
\label{fig:preimage_and_image_x}
\end{center}

\begin{inner_problem}
\item The points in the preimage are discontinuous, but they belong to a particular, infinite set of points. Write the equation of that set. (Hint: What is $y$ in terms of $x$?)
\end{inner_problem}

They belong to a line! More precisely, they are all on the line $y=x-1$.

\begin{inner_problem}
\item Write an equation for the image of that set.
\end{inner_problem}

It appears the equation of the image is simply $y=-1$. We can verify this by multiplying $M$ by $\begin{bmatrix} t+1 \\ t \end{bmatrix}$:

$$\begin{bmatrix} 2 & 3 \\ -1 & 1 \end{bmatrix} \begin{bmatrix} t+1 \\ t \end{bmatrix} = \begin{bmatrix} 2(t+1) + 3t \\ -(t+1) + t \end{bmatrix} = \begin{bmatrix} 5t + 2 \\ -1 \end{bmatrix}.$$

Indeed, the $y$ coordinate of the image is $-1$, and the $x$ coordinate can be any real number.

\begin{inner_problem}
\item What other characteristic of the preimage points also applies to the image?
\end{inner_problem}

The points in the preimage are equidistant, taken as consecutive pairs. This is also true of the image.

\begin{inner_problem}
\item Name two things that seem to be conserved when mapping points with a matrix.
\end{inner_problem}

It seems (i) collinearity and (ii) equally spaced points on a line have their characteristics preserved. Note that not \textit{all} equidistant points will have this property conserved. Think back to the first problem, for example, where the pairs of points $((0,0),(1,0))$ and $((0,0),(0,1))$ started off equidistant, but ended up not equidistant. Indeed, they have to be collinear for this to be preserved.

\begin{outer_problem}
\item
\end{outer_problem}

\begin{inner_problem}[start=1]
\item Choose a different $2\times 2$ matrix and a different set of three collinear, equally spaced unique points. Perform the appropriate matrix multiplication.
\end{inner_problem}

I'm gonna choose the transformation matrix $M = \begin{bmatrix} 3 & -2 \\ 2 & 3 \end{bmatrix}$, and the points $\begin{bmatrix} 2 & 4 & 6 \\ -1 & 0 & 1 \end{bmatrix}$. The multiplication is straightforward:

$$\begin{bmatrix} 3 & -2 \\ 2 & 3 \end{bmatrix} \begin{bmatrix} 2 & 4 & 6 \\ -1 & 0 & 1 \end{bmatrix} = \begin{bmatrix} 8 & 12 & 16 \\ 1 & 8 & 15 \end{bmatrix}.$$

\begin{inner_problem}
\item Graph and label the preimage points and the image points.
\end{inner_problem}

The picture is shown in Figure~\ref{fig:preimage_and_image_y}.

\begin{center}
\begin{asy}[width=0.45\textwidth]
import graph;

pair A = (2,-1);
pair B = (4,0);
pair C = (6,1);

pair Ap = (8,1);
pair Bp = (12,8);
pair Cp = (16,15);

dot(A);
dot(B);
dot(C);

dot(Ap);
dot(Bp);
dot(Cp);

label("$A$", A, S);
label("$B$", B, N);
label("$C$", C, N);

label("$A'$", Ap, SE);
label("$B'$", Bp, S);
label("$C'$", Cp, S);

draw((1.4 * C - 0.4 * A) -- (1.2 * A - 0.2 * C), dashed);
draw((1.2 * Cp - 0.2 * Ap) -- (1.4 * Ap - 0.4 * Cp), dashed);

xaxis("$x$");
yaxis("$y$");
\end{asy}
\captionof{figure}{The preimage and image.}
\label{fig:preimage_and_image_y}
\end{center}

\begin{inner_problem}
\item Have the collinearity and equal spacing been preserved?
\end{inner_problem}

Indeed! The vector from $(8,1)$ to $(12,8)$ is $\langle 4, 7\rangle$ and the vector from $(12,8)$ to $(16,15)$ is also $\langle 4, 7\rangle$.

\begin{inner_problem}
\item Make a conjecture about when a matrix will preserve collinearity and when a matrix will preserve equal spacing.
\end{inner_problem}

Since two random matrices have both done it, we conjecture that all matrices do so.

\begin{outer_problem}
\item Now, we will check your conjecture.
\end{outer_problem}

\begin{inner_problem}[start=1]
\item Start with a general $2\times 2$ matrix and three equally spaced points on a line, and multiply the two matrices:
$$\left[\begin{array}{cc}a & b \\ c & d\end{array}\right]
\left[\begin{array}{ccc}x-h & x & x+h \\ m(x-h)+k & mx+k & m(x+h)+k\end{array}\right]=
\left[\begin{array}{ccc}\phantom{0} & \phantom{0} & \phantom{0} \\ \phantom{0} \end{array}\right].$$
\end{inner_problem}

$$\begin{bmatrix} a & b \\ c & d \end{bmatrix} \begin{bmatrix} x-h & x & x+h \\ m(x-h)+k & mx+k & m(x+h)+k \end{bmatrix}$$

$$ = \begin{bmatrix} a(x-h) + b(m(x-h)+k) & ax + b(mx+k) & a(x+h) + b(m(x+h)+k) \\ c(x-h)+d(m(x-h)+k) & cx + d(mx+k) & c(x+h) + d(m(x+h)+k) \end{bmatrix}.$$

\begin{inner_problem}
\item How do you know that the second matrix indeed represents collinear and equally spaced points?
\end{inner_problem}

There's a couple of ways to rationalize this, but my favorite is to simply consider the vector between consecutive points.

Our second matrix is $\begin{bmatrix} x-h & x & x+h \\ m(x-h)+k & mx+k & m(x+h)+k \end{bmatrix}$. The vector from the first point to the second is $\langle h, mh \rangle$; similarly, the vector from the second point to the third is $\langle h, mh \rangle$. This means the points are collinear, because the vectors between each pair have the same direction, and equidistant, because each consecutive pair has the same distance.

\begin{inner_problem}
\item Are there any sets of collinear points that aren't representable by the $2\times 3$ matrix?
\end{inner_problem}

Yes there are. We cannot represent collinear points that go directly vertically, because (informally) that would mean $h=0$ and $m=\infty$. More precisely, we must have $h=0$, but then there is no real $m$ such that $m(x-h)+k\neq mx+k \neq m(x+h)+k$. As an example, we cannot represent the points $$\begin{bmatrix} 1 & 1 & 1 \\ -50 & 0 & 50 \end{bmatrix}.$$

\begin{inner_problem}
\item Are the points in the image collinear? Show why or why not.
\end{inner_problem}

Yes they are. Again, we think about the vector between consecutive points: for the first pair it's

\begin{align*}
V_1 &= \langle (ax+b(mx+k)) - (a(x-h) + b(m(x-h)+k)), (cx + d(mx+k)) - (c(x-h)+d(m(x-h)+k)) \rangle \\
&= \langle ah + bmh, ch + dmh \rangle;
\end{align*}

for the second pair it's

\begin{align*}
V_2 &= \langle (a(x+h) + b(m(x+h)+k)) - (ax + b(mx+k)), (c(x+h) + d(m(x+h)+k)) - (cx + d(mx+k)) \rangle \\
&= \langle ah + bmh, ch + dmh \rangle = V_1.
\end{align*}

Since $V_2=V_1$, the points are collinear (and equidistant).

\begin{inner_problem}
\item Can you find values for $a$, $b$, $c$, and $d$ so that the image does not lie on a unique line? (Hint: all of the points in the image must lie on no line, or on multiple lines.)
\end{inner_problem}

At first this seems like it's contradicting our conjecture, but the devil's in the details. If all the points are coincident---that is, they're all equal---then there are infinitely many lines passing through it.

We set $a=b=c=d=0$ so that $M=\begin{bmatrix} 0 & 0 \\ 0 & 0 \end{bmatrix}$, which maps every point to $(0,0)$. Then there are infinitely many lines going through the image.

\begin{inner_problem}
\item Use the distance formula---or some other justification---to answer whether the points in the image are equally spaced.
\end{inner_problem}

We showed that they're equidistant two subproblems ago already, with vectors. But I'll also show it with the distance formula since that what the question suggests.

The distance between the first two points is

$$\sqrt{((ax+b(mx+k)) - (a(x-h) + b(m(x-h)+k)))^2 + ((cx + d(mx+k)) - (c(x-h)+d(m(x-h)+k)))^2}$$
$$=\sqrt{(ah+bmh)^2 + (ch+dmh)^2}.$$

Looks familiar.... The distance between the second points is
$$\sqrt{((a(x+h) + b(m(x+h)+k)) - (ax + b(mx+k)))^2 + ((c(x+h) + d(m(x+h)+k)) - (cx + d(mx+k)))^2}$$
$$=\sqrt{(ah+bmh)^2 + (ch+dmh)^2}.$$

The distances are equal; they are equidistant.

\begin{outer_problem}
\item There is a point which remains \textbf{fixed}---its image is the same as its preimage---when multiplied by the matrix $\left[\begin{array}{cc}2 & 3 \\ 4 & 5 \end{array}\right]$. That is, $\left[\begin{array}{cc}2 & 3 \\ 4 & 5 \end{array}\right]\left[\begin{array}{c} x \\ y \end{array}\right]=\left[\begin{array}{c} x \\ y \end{array}\right]$.
\end{outer_problem}

\begin{inner_problem}[start=1]
\item Solve the above matrix equation for $x$ and $y$ to find the point.
\end{inner_problem}

We multiply out the right side:

$$\begin{bmatrix}2 & 3 \\ 4 & 5 \end{bmatrix} \begin{bmatrix} x \\ y \end{bmatrix} = \begin{bmatrix}2x + 3y \\ 4x + 5y \end{bmatrix}.$$

Equating corresponding entries, we get the system of equations

$$\begin{cases} x = 2x + 3y \\ y = 4x + 5y \end{cases}.$$

Solving this system gives $(x,y)=(0,0)$. How mundane....

\begin{inner_problem}
\item There is a point $Q=\begin{bsmallmatrix}e \\ f \end{bsmallmatrix}$ that remains fixed no matter what matrix you multiply it by. Can you guess what point that is?
\end{inner_problem}

Looks like the point is $Q=\begin{bmatrix} 0 \\ 0 \end{bmatrix}$, given the answer to the previous subproblem.

\begin{inner_problem}
\item Prove your conjecture by plugging your point $Q$ into $\left[\begin{array}{cc}a & b \\ c & d\end{array}\right]Q=Q$.
\end{inner_problem}

We do the multiplication:

$$\begin{bmatrix} a & b \\ c & d \end{bmatrix} \begin{bmatrix} 0 \\ 0 \end{bmatrix} = \begin{bmatrix} 0 \\ 0 \end{bmatrix} = Q.$$

Indeed, $Q$ always remains fixed.

\begin{outer_problem}
\item Begin with a triangle with vertices $(5,0)$, $(10,0)$, and $(5,10)$.
\end{outer_problem}

\begin{inner_problem}[start=1]
\item Map the vertices with the following matrices.
\end{inner_problem}

\begin{iinner_problem}[start=1]
\item $\left[\begin{array}{cc}1 & 0 \\ 0 & 1 \end{array}\right]$
\end{iinner_problem}

We multiply them:

$$\begin{bmatrix} 1 & 0 \\ 0 & 1 \end{bmatrix} \begin{bmatrix} 5 & 10 & 5 \\ 0 & 0 & 10 \end{bmatrix} = \begin{bmatrix} 5 & 10 & 5 \\ 0 & 0 & 10 \end{bmatrix}.$$

\begin{iinner_problem}
\item $\left[\begin{array}{cc}.6 & -.8 \\ .8 & .6 \end{array}\right]$
\end{iinner_problem}

$$\begin{bmatrix} .6 & -.8 \\ .8 & .6 \end{bmatrix} \begin{bmatrix} 5 & 10 & 5 \\ 0 & 0 & 10 \end{bmatrix} = \begin{bmatrix} 3 & 6 & -5 \\ 4 & 8 & 10 \end{bmatrix}.$$

\begin{iinner_problem}
\item $\left[\begin{array}{cc}.6 & .8 \\ .8 & -.6 \end{array}\right]$
\end{iinner_problem}

$$\begin{bmatrix} .6 & .8 \\ .8 & -.6 \end{bmatrix} \begin{bmatrix} 5 & 10 & 5 \\ 0 & 0 & 10 \end{bmatrix} = \begin{bmatrix} 3 & 6 & 11 \\ 4 & 8 & -2 \end{bmatrix}.$$

\begin{inner_problem}
\item Why will the new triangle defined by these vertices be the image of the starting triangle?
\end{inner_problem}

We only transformed the vertices, but because matrix multiplication is a linear transformation, lines are mapped to lines. Thus, the new triangle defined by these vertices is the image of the old triangle, in the sense that the sides of the original are mapped onto this new triangle as well.

\begin{inner_problem}
\item Accurately graph the preimage, then the image for each matrix on three separate sets of axes.
\end{inner_problem}

(i) is shown in Figure~\ref{fig:preimage_and_image_i}; (ii) is shown in Figure~\ref{fig:preimage_and_image_ii}; (iii) is shown in Figure~\ref{fig:preimage_and_image_iii}.

\begin{minipage}{0.49\textwidth}
\begin{asy}[width=0.9\textwidth]
import graph;

pair A = (5, 0);
pair B = (10,0);
pair C = (5,10);

pair Ap = A;
pair Bp = B;
pair Cp = C;

dot(A);
dot(B);
dot(C);

label("$A=A'$", A, 1.5*NW);
label("$B=B'$", B, NE);
label("$C=C'$", C, N);

xaxis("$x$", Bottom, LeftTicks, xmin=0);
yaxis("$y$", Left, RightTicks);

draw(A--B--C--cycle,dashed);
\end{asy}
\captionof{figure}{Preimage and image for (i).}
\label{fig:preimage_and_image_i}
\end{minipage}\hfill
\begin{minipage}{0.49\textwidth}
\begin{asy}[width=0.9\textwidth]
import graph;

pair A = (5, 0);
pair B = (10,0);
pair C = (5,10);

pair Ap = (3,4);
pair Bp = (6,8);
pair Cp = (-5,10);

dot(A);
dot(B);
dot(C);

dot(Ap);
dot(Bp);
dot(Cp);

label("$A$", A, 1.5*NW);
label("$B$", B, NE);
label("$C$", C, N);

label("$A'$", Ap, SW);
label("$B'$", Bp, NE);
label("$C'$", Cp, NW);

draw(A--B--C--cycle,dashed);
draw(Ap--Bp--Cp--cycle,dashed);

xaxis("$x$", Bottom, LeftTicks, xmin=-7);
yaxis("$y$", Left, RightTicks);
\end{asy}
\captionof{figure}{Preimage and image for (ii).}
\label{fig:preimage_and_image_ii}
\end{minipage}
\begin{center}
\begin{asy}[width=0.44\textwidth]
import graph;

pair A = (5, 0);
pair B = (10,0);
pair C = (5,10);

pair Ap = (3,4);
pair Bp = (6,8);
pair Cp = (11,-2);

dot(A);
dot(B);
dot(C);

dot(Ap);
dot(Bp);
dot(Cp);

label("$A$", A, 1.5*NW);
label("$B$", B, NE);
label("$C$", C, N);

label("$A'$", Ap, SW);
label("$B'$", Bp, NE);
label("$C'$", Cp, S);

draw(A--B--C--cycle,dashed);
draw(Ap--Bp--Cp--cycle,dashed);

xaxis("$x$", YEquals(-4), LeftTicks, xmin=0);
yaxis("$y$", Left, RightTicks, ymin=-4);
\end{asy}
\captionof{figure}{Preimage and image for (iii).}
\label{fig:preimage_and_image_iii}
\end{center}

\begin{inner_problem}
\item For each, describe the transformation as fully as you can. Try to classify them on the transformations we mentioned earlier, and quantify them if necessary (e.g. to describe the line of reflection or angle of rotation).
\end{inner_problem}

(i) is the identity transformation; it does absolutely nothing. Indeed, the matrix $I_2=\begin{bmatrix} 1 & 0 \\ 0 & 1 \end{bmatrix}$ is called the identity matrix of size $2$.

(ii) is a rotation about the origin by $\tan^{-1} \frac{4}{3} \approx 53.13^\circ$. This is found by taking $\tan^{-1} \frac{y}{x}$ for the point $(x,y)=A'=(3,4)$.

(iii) is a reflection about the line $y = \frac{x}{2}$. The easiest way to see this is that because the origin is fixed, the reflection line must pass through $(0,0)$, so it is $y =mx$ for some real number $m$. But it also must pass through the midpoint of $\overline{AA'}$ (and $\overline{BB'}$, $\overline{CC'}$), which is $(4,2)$. Thus, $m=\frac{2}{4}=\frac{1}{2}$.

\begin{outer_problem}
\item Soon, we will map the unit square: it has vertices $(0,0)$, $(1,0)$, $(0,1)$, and $(1,1)$. We could actually get the entire image from the image of the unit vectors $(1,0)$ and $(0,1)$, which will be useful later.
\end{outer_problem}

\begin{inner_problem}[start=1]
\item How can we obtain the image of $(1,1)$ from the images of $(1,0)$ and $(0,1)$?
\end{inner_problem}

We can use the associative property of matrix multiplication to help us! If the image of a point $P$ is $i_P$, then we have

$$i_{(1,1)} = M\begin{bmatrix} 1 \\ 1 \end{bmatrix} = M\left(\begin{bmatrix} 1 \\ 0 \end{bmatrix} + \begin{bmatrix} 0 \\ 1 \end{bmatrix}\right) = M\begin{bmatrix} 1 \\ 0 \end{bmatrix} + M\begin{bmatrix} 0 \\ 1 \end{bmatrix} = i_{(1,0)} + i_{(0,1)}.$$

In words, we sum the images of $(1,0)$ and $(0,1)$ to get the image of $(1,1)$. Neat!

\begin{inner_problem}
\item Of $(0,0)$?
\end{inner_problem}

This is sort of a trick question. We have $i_{(0,0)} = M\begin{bmatrix} 0 \\ 0 \end{bmatrix} = \begin{bmatrix} 0 \\ 0 \end{bmatrix},$ so the image is always $(0,0)$.

\begin{outer_problem}
\item
\end{outer_problem}

\begin{inner_problem}[start=1]
\item Take the matrix $\left[\begin{smallmatrix}1 & 2 \\ 0 & 1\end{smallmatrix}\right]$ and see what it does to the unit square. Please graph this, being careful to label each point and its image. The multiplication is done for you below.

$$\begin{blockarray}{cccccc}
& & A & B & C & D \\
\begin{block}{[cc][cccc]}
1 & 2 & 0 & 1 & 1 & 0 \\
0 & 1 & 0 & 0 & 1 & 1 \\
\end{block}
\end{blockarray} =
\begin{blockarray}{cccc}
A' & B' & C' & D' \\
\begin{block}{[cccc]}
0 & 1 & 3 & 2 \\
0 & 0 & 1 & 1 \\
\end{block}
\end{blockarray}.$$
\end{inner_problem}

\begin{asydef}
import graph;

void drawUnitSquareTransform(real a, real b, real c, real d, pair dirA, pair dirB, pair dirC, pair dirD) {
	pair A = (0,0);
	pair B = (1,0);
	pair C = (1,1);
	pair D = (0,1);

	pair Ap = (0,0);
	pair Bp = (a,c);
	pair Dp = (b,d);
	pair Cp = Bp+Dp;

	draw(A--B--C--D--A,dashed);
	draw(Ap--Bp--Cp--Dp--Ap);

	dot(Ap);
	dot(Bp);
	dot(Cp);
	dot(Dp);

	pair[] original = {A,B,C,D};
	pair[] image = {Ap,Bp,Cp,Dp};

	pair[] original_dirs = {SW, SE, NE, NW};
	pair[] image_dirs = {unit(unit(Ap-Dp)+unit(Ap-Bp)),
	unit(unit(Bp-Cp)+unit(Bp-Ap)),
	unit(unit(Cp-Dp)+unit(Cp-Bp)),
	unit(unit(Dp-Cp)+unit(Dp-Ap))};

	string[] original_names = {"A","B","C","D"};
	int[] images_labeled_already = {0,0,0,0};

	for (int i = 0; i < 4; ++i) {
		pair original_p = original[i];
		pair original_d = original_dirs[i];
		string original_n = original_names[i];

		string[] extra_names = {};

		for (int j = 0; j < 4; ++j) {
			if (length(image[j]-original[i]) < 0.001) { // Good enough
				extra_names.push(original_names[j] + "'");

				images_labeled_already[j] = 1;
			}
		}

		string label_str = "$" + original_n;

		for (int j = 0; j < extra_names.length; ++j) {
			label_str += "=" + extra_names[j];
		}

		label_str += "$";

		label(label_str, original_p, original_d);
	}

	for (int i = 0; i < 4; ++i) {
		if (images_labeled_already[i] == 1) continue;

		string extra = "";

		for (int j = i+1; j < 4; ++j) {
			if (images_labeled_already[j] == 1) continue;

			if (length(image[j]-image[i]) < 0.001) {
				extra += "= " + original_names[j] + "'";
				images_labeled_already[j] = 1;
			}
		}

		label("$" + original_names[i] + "'" + extra + "$", image[i], image_dirs[i]);
	}

	xaxis();
	yaxis();
}
\end{asydef}

\begin{center}
\begin{asy}[width=0.5\textwidth]
drawUnitSquareTransform(1,2,0,1,N,N,NE,NW);
\end{asy}
\end{center}

\begin{inner_problem}
\item What happens to the area of the image versus the preimage?
\end{inner_problem}

They are equal. After all, both are parallelograms with base $1$ and height $1$.

\begin{inner_problem}
\item We have $AB=BC$, but is $A'B'$ equal to $B'C'$? Should it?
\end{inner_problem}

No, $A'B' \neq B'C'$. It doesn't have to, because though the points are equidistant, they are not collinear, and don't have to be equidistant in the image.

\begin{outer_problem}
\item
\end{outer_problem}

\begin{inner_problem}[start=1]
\item When is the ratio of distances between points in the image the same as in the preimage?
\end{inner_problem}

This is true when the image is a dilation, reflection, or rotation: or combination of the three.

\begin{inner_problem}
\item What is the image of the origin under any matrix mapping?
\end{inner_problem}

The image of $(0,0)$ is always $(0,0)$.

\begin{inner_problem}
\item What are the images of the points $(1,0)$ and $(0,1)$ under the mapping $\begin{blockarray}{[cc]} a & b \\ c & d \end{blockarray}$?
\end{inner_problem}

The image of $(1,0)$ is just $(a,c)$. The image of $(0,1)$ is $(b,d)$.

\begin{inner_problem}
\item Knowing the images of $(1,0)$ and $(0,1)$, how do we find the image of $(1,1)$ algebraically and geometrically?
\end{inner_problem}

The image of $(1,1)$, as we found earlier, is the sum of the images of $(1,0)$ and $(0,1)$. Thus, it is $(a+b,c+d)$. Geometrically, it forms a parallelogram with the images of $(1,0)$ and $(0,1)$.

\begin{outer_problem}
\item How do these matrices map the plane? For each mapping, write a matrix for the images of the four corners of the unit square, then graph the preimage and image. Describe the mapping using words from geometry such as congruent, similar, rotate, reflect, shear, stretch, magnitude, and direction. \label{prob:map_plane_sixteen_matrices}
\end{outer_problem}

\newcommand{\mtrxtbt}[4] {$\left[\begin{array}{cc}#1 & #2 \\ #3 & #4 \end{array}\right]$}
\begin{inner_problem}[start=1]
\item \mtrxtbt{1}{0}{0}{-1}
\end{inner_problem}

Image matrix: $\begin{bmatrix} 0 & 1 & 1 & 0 \\ 0 & 0 & -1 & -1 \end{bmatrix}$.

\begin{center}
\begin{asy}
size(150);
drawUnitSquareTransform(1,0,0,-1,N,N,N,N);
\end{asy}
\end{center}

This is a reflection over the $x$-axis.

\begin{inner_problem}
\item \mtrxtbt{-1}{0}{0}{-1}
\end{inner_problem}

Image matrix: $\begin{bmatrix} 0 & -1 & -1 & 0 \\ 0 & 0 & -1 & -1 \end{bmatrix}$.

\begin{center}
\begin{asy}
size(150);
drawUnitSquareTransform(-1,0,0,-1,N,N,N,N);
\end{asy}
\end{center}

This is a reflection about the origin $(0,0)$.

\begin{inner_problem}
\item \mtrxtbt{2}{0}{0}{2}
\end{inner_problem}

Image matrix: $\begin{bmatrix} 0 & 2 & 2 & 0 \\ 0 & 0 & 2 & 2 \end{bmatrix}$.

\begin{center}
\begin{asy}
size(150);
drawUnitSquareTransform(2,0,0,2,N,N,N,N);
\end{asy}
\end{center}

This is a dilation about the origin $(0,0)$ by a factor of $2$.

\begin{inner_problem}
\item \mtrxtbt{0}{1}{-1}{0}
\end{inner_problem}

Image matrix: $\begin{bmatrix} 0 & 0 & 1 & 1 \\ 0 & -1 & -1 & 0 \end{bmatrix}$.

\begin{center}
\begin{asy}
size(150);
drawUnitSquareTransform(0,1,-1,0,N,N,N,N);
\end{asy}
\end{center}

This is a rotation by $90^\circ=\frac{\pi}{2}$ clockwise, or $270^\circ=\frac{3\pi}{2}$ counterclockwise.

\begin{inner_problem}
\item \mtrxtbt{0}{1}{1}{0}
\end{inner_problem}

Image matrix: $\begin{bmatrix} 0 & 0 & 1 & 1 \\ 0 & 1 & 1 & 0 \end{bmatrix}$.

\begin{center}
\begin{asy}
size(150);
draw((-0.5,-0.5)--(1.5,1.5), dashed);
drawUnitSquareTransform(0,1,1,0,N,N,N,N);
\end{asy}
\end{center}

This is a reflection about the line $y=x$, which is marked in the diagram.

\begin{inner_problem}
\item \mtrxtbt{0}{0}{0}{0}
\end{inner_problem}

Image matrix: $\begin{bmatrix} 0 & 0 & 0 & 0 \\ 0 & 0 & 0 & 0 \end{bmatrix}$.

\begin{center}
\begin{asy}
size(150);
drawUnitSquareTransform(0,0,0,0,N,N,N,N);
\end{asy}
\end{center}

This is a projection onto the point $(0,0)$. Every point goes to $(0,0)$!

\begin{inner_problem}
\item \mtrxtbt{1}{0}{0}{1}
\end{inner_problem}

Image matrix: $\begin{bmatrix} 0 & 1 & 1 & 0 \\ 0 & 0 & 1 & 1 \end{bmatrix}$.

\begin{center}
\begin{asy}
size(150);
drawUnitSquareTransform(1,0,0,1,N,N,N,N);
\end{asy}
\end{center}

This is the identity transformation: it does nothing. Every point is mapped to itself.

\begin{inner_problem}
\item \mtrxtbt{3}{0}{0}{1}
\end{inner_problem}

Image matrix: $\begin{bmatrix} 0 & 3 & 3 & 0 \\ 0 & 0 & 1 & 1 \end{bmatrix}$.

\begin{center}
\begin{asy}
size(150);
drawUnitSquareTransform(3,0,0,1,N,N,N,N);
\end{asy}
\end{center}

This is a stretch by a factor of $3$ along the $x$-axis.

\begin{inner_problem}
\item \mtrxtbt{1}{0}{-3}{1}
\end{inner_problem}

Image matrix: $\begin{bmatrix} 0 & 1 & 1 & 0 \\ 0 & -3 & -2 & 1 \end{bmatrix}$.

\begin{center}
\begin{asy}
size(150);
drawUnitSquareTransform(1,0,-3,1,N,N,N,N);
\end{asy}
\end{center}

This is a shear by a factor of $3$ along the negative $y$-axis.

\begin{inner_problem}
\item \mtrxtbt{2}{2}{-3}{-3}
\end{inner_problem}

Image matrix: $\begin{bmatrix} 0 & 2 & 4 & 2 \\ 0 & -3 & -6 & -3 \end{bmatrix}$.

\begin{center}
\begin{asy}
size(150);
drawUnitSquareTransform(2,2,-3,-3,N,N,N,N);
\end{asy}
\end{center}

This is a mapping onto the line $-3x=2y\Longrightarrow 3x+2y=0$. Note that this isn't a \textit{projection}, since that has extra constraints.

\begin{inner_problem}
\item \mtrxtbt{3}{2}{4}{-1}
\end{inner_problem}

Image matrix: $\begin{bmatrix} 0 & 3 & 5 & 2 \\ 0 & 4 & 3 & -1 \end{bmatrix}$.

\begin{center}
\begin{asy}
size(150);
drawUnitSquareTransform(3,2,4,-1,N,N,N,N);
\end{asy}
\end{center}

This is a rather complex transformation. We'll learn later how to tackle this properly. If you're curious, this matrix is the following operations in order:

\begin{itemize}
\item Shear by $\frac{2}{3}$ along the (positive) $x$-axis
\item Stretch by $-\frac{11}{3}$ along the $y$-axis
\item Shear by $4$ along the $y$-axis
\item Stretch by $3$ along the $x$-axis.
\end{itemize}

\begin{inner_problem}
\item \mtrxtbt{\frac{\sqrt{2}}{2}}{\frac{\sqrt{2}}{2}}{\frac{\sqrt{2}}{2}}{-\frac{\sqrt{2}}{2}}
\end{inner_problem}

Image matrix: $\begin{bmatrix} 0 & \frac{\sqrt{2}}{2} & \sqrt{2} & \frac{\sqrt{2}}{2} \\ 0 & \frac{\sqrt{2}}{2} & 0 & -\frac{\sqrt{2}}{2} \end{bmatrix}$.

\begin{center}
\begin{asy}
size(200);
real sq2f = sqrt(2)/2;
pair St = (-1,-sqrt(2)+1);
pair Et = -2*St;

draw(St--Et, dotted);
label("$\theta=\frac{\pi}{8}$", Et, NE);
drawUnitSquareTransform(sq2f,sq2f,sq2f,-sq2f,N,N,N,N);
\end{asy}
\end{center}

This a reflection over the line $\theta=22.5^\circ=\frac{\pi}{8}$, or $y=(\sqrt{2}-1)x$.

\begin{inner_problem}
\item \mtrxtbt{\frac{\sqrt{2}}{2}}{\frac{\sqrt{2}}{2}}{-\frac{\sqrt{2}}{2}}{\frac{\sqrt{2}}{2}}
\end{inner_problem}

Image matrix: $\begin{bmatrix} 0 & \frac{\sqrt{2}}{2} & \sqrt{2} & \frac{\sqrt{2}}{2} \\ 0 & -\frac{\sqrt{2}}{2} & 0 & \frac{\sqrt{2}}{2} \end{bmatrix}$.

\begin{center}
\begin{asy}
size(150);
real sq2f = sqrt(2)/2;
drawUnitSquareTransform(sq2f,sq2f,-sq2f,sq2f,N,N,N,N);
\end{asy}
\end{center}

This is a rotation by $45^\circ=\frac{\pi}{4}$ clockwise about the origin, or a rotation of $315^\circ = \frac{7\pi}{4}$ counterclockwise about the origin.

\begin{inner_problem}
\item \mtrxtbt{\frac{\sqrt{3}}{2}}{\frac{1}{2}}{\frac{1}{2}}{-\frac{\sqrt{3}}{2}}
\end{inner_problem}

Image matrix: $\begin{bmatrix} 0 & \frac{\sqrt{3}}{2} & \frac{1+\sqrt{3}}{2} & \frac{1}{2} \\ 0 & \frac{1}{2} & \frac{1-\sqrt{3}}{2} & -\frac{\sqrt{3}}{2} \end{bmatrix}$.

\begin{center}
\begin{asy}
size(150);
pair St = (-1,sqrt(3)-2);
pair Et = -1.5*St;

draw(St--Et, dotted);
label("$\theta=\frac{\pi}{12}$", Et, NE);
drawUnitSquareTransform(sqrt(3)/2, 1/2, 1/2, -sqrt(3)/2,N,N,N,N);
\end{asy}
\end{center}

This is a reflection about the line $\theta = 15^\circ = \frac{\pi}{12}$, or the line $y=(2-\sqrt{3})x$.

\begin{inner_problem}
\item \mtrxtbt{\frac{\sqrt{3}}{2}}{-\frac{1}{2}}{\frac{1}{2}}{\frac{\sqrt{3}}{2}}
\end{inner_problem}

Image matrix: $\begin{bmatrix} 0 & \frac{\sqrt{3}}{2} & \frac{\sqrt{3}-1}{2} & -\frac{1}{2} \\ 0 & \frac{1}{2} & \frac{1+\sqrt{3}}{2} & \frac{\sqrt{3}}{2} \end{bmatrix}$.

\begin{center}
\begin{asy}
size(150);
drawUnitSquareTransform(sqrt(3)/2, -1/2, 1/2, sqrt(3)/2,N,N,N,N);
\end{asy}
\end{center}

This is a rotation of $30^\circ = \frac{\pi}{6}$ counterclockwise about the origin.

\begin{inner_problem}
\item \mtrxtbt{\frac{\sqrt{3}}{2}}{\frac{1}{2}}{-\frac{1}{2}}{\frac{\sqrt{3}}{2}}
\end{inner_problem}

Image matrix: $\begin{bmatrix} 0 & \frac{\sqrt{3}}{2} & \frac{1+\sqrt{3}}{2} & \frac{1}{2} \\ 0 & -\frac{1}{2} & \frac{\sqrt{3}-1}{2} & \frac{\sqrt{3}}{2} \end{bmatrix}$.

\begin{center}
\begin{asy}
size(150);
drawUnitSquareTransform(sqrt(3)/2, 1/2, -1/2, sqrt(3)/2,N,N,N,N);
\end{asy}
\end{center}

This is rotation of $30^\circ = \frac{\pi}{6}$ \textit{clockwise} about the origin, or $330^\circ = \frac{11\pi}{6}$ counterclockwise about the origin.

\begin{outer_problem}
\item Carry out the following multiplications and convince yourself they are equivalent mappings of the $x$ and $y$ coordinates.
\end{outer_problem}

\begin{inner_problem}[start=1]
\item $\left[\begin{array}{cc}a & b \\ c & d\end{array}\right]\left[\begin{array}{c} u \\ v \end{array}\right]=\left[\begin{array}{c} \phantom{u} \\ \phantom{v} \end{array}\right]$ $\phantom{\begin{array}{c}u \\ v \\ 1 \end{array}}$
\end{inner_problem}

This is simple:

$$\begin{bmatrix} a & b \\ c & d \end{bmatrix} \begin{bmatrix} u \\ v \end{bmatrix} = \begin{bmatrix} au+bv \\ cu+dv \end{bmatrix}.$$

\begin{inner_problem}
\item $\left[\begin{array}{ccc}a & b & 0 \\ c & d & 0 \\ 0 & 0 & 1 \end{array}\right]\left[\begin{array}{c}u \\ v \\ 1 \end{array}\right] = \left[\begin{array}{c}\phantom{u} \\ \phantom{v} \\ \phantom{1} \end{array}\right]$
\end{inner_problem}

$$\begin{bmatrix} a & b & 0 \\ c & d & 0 \\ 0 & 0 & 1 \end{bmatrix} \begin{bmatrix} u \\ v \\ 1 \end{bmatrix} = \begin{bmatrix} au + bv + 0\cdot 1 \\ cu + dv + 0\cdot 1 \\ 0u + 0v + 0\cdot 1 \end{bmatrix} \begin{bmatrix} au+bv \\ cu+dv \\ 1 \end{bmatrix}.$$

They are equivalent because the $x,y$ components of each are the same, and $z$ is unchanged in the second equation.

\begin{outer_problem}
\item
\end{outer_problem}

\begin{inner_problem}[start=1]
\item Multiply these matrices: $\left[\begin{array}{ccc} 1 & 0 & \alpha \\ 0 & 1 & \beta \\ 0 & 0 & 1 \end{array}\right]\left[\begin{array}{c}u \\ v \\ 1 \end{array}\right]=\left[\begin{array}{c}\phantom{u} \\ \phantom{v} \\ \phantom{1}\end{array}\right].$ \label{prob:translation_matrix}
\end{inner_problem}

$$\begin{bmatrix} 1 & 0 & \alpha \\ 0 & 1 & \beta \\ 0 & 0 & 1 \end{bmatrix} \begin{bmatrix}u \\ v \\ 1 \end{bmatrix} = \begin{bmatrix} u + \alpha \\ v + \beta \\ 1 \end{bmatrix}.$$

\begin{inner_problem}
\item Fill in the blanks: The result of the above multiplication is that the point $(u,v,1)$ has been translated by $\underline{\phantom{0000}}$ in the $x$ direction, $\underline{\phantom{0000}}$ in the $y$ direction, and is still anchored to the plane $z=\underline{\phantom{000}}$.
\end{inner_problem}

The result of the above multiplication is that the point $(u,v,1)$ has been translated by \underline{$\alpha$} in the $x$ direction, \underline{$\beta$} in the $y$ direction, and is still anchored to the plane $z=$\underline{$1$}.

\begin{outer_problem}
\item
\end{outer_problem}

\begin{inner_problem}[start=1]
\item Write a matrix which translates a point $(x,y,1)$ $4$ units in the $x$ direction and $7$ units in the $y$ direction, leaving $z$ fixed at $1$.
\end{inner_problem}

We have $\alpha = 4$ and $\beta = 7$:

$$\begin{bmatrix} 1 & 0 & 4 \\ 0 & 1 & 7 \\ 0 & 0 & 1 \end{bmatrix} \begin{bmatrix} x \\ y \\ 1 \end{bmatrix} = \begin{bmatrix} x + 4 \\ y + 7 \\ 1 \end{bmatrix}.$$

\begin{inner_problem}
\item Check your work by applying your matrix to the point $(3,5,1)$.
\end{inner_problem}

The expected result is $(3+4,5+7,1)=(7,12,1)$. We multiply:

$$\begin{bmatrix} 1 & 0 & 4 \\ 0 & 1 & 7 \\ 0 & 0 & 1 \end{bmatrix} \begin{bmatrix} 3 \\ 5 \\ 1 \end{bmatrix} = \begin{bmatrix} 7 \\ 12 \\ 1 \end{bmatrix}.$$

Looks good.

\begin{outer_problem}
\item Do these two multiplications. What does each represent?
\end{outer_problem}

\begin{inner_problem}[start=1]
\item $\left[\begin{array}{ccc}a & b & 0 \\ c & d & 0 \\ 0 & 0 & 1 \end{array}\right]\left[\begin{array}{ccc} 1 & 0 & \alpha \\ 0 & 1 & \beta \\ 0 & 0 & 1 \end{array}\right]$
\end{inner_problem}

We multiply:

$$\begin{bmatrix} a & b & 0 \\ c & d & 0 \\ 0 & 0 & 1 \end{bmatrix} \begin{bmatrix} 1 & 0 & \alpha \\ 0 & 1 & \beta \\ 0 & 0 & 1 \end{bmatrix} = \begin{bmatrix} a & b & a\alpha + b\beta \\ c & d & c\alpha + d\beta \\ 0 & 0 & 1 \end{bmatrix}$$

Because we have chosen the column vector to represent points, the transformations take place from right to left. Thus, the right matrix is the first transformation, and the left matrix is the second transformation. Therefore, this multiplication represents a translation by $\langle\alpha,\beta,0\rangle$, then a 2D matrix transformation by $\left[\begin{smallmatrix} a & b \\ c & d \end{smallmatrix}\right]$.

\begin{inner_problem}
\item $\left[\begin{array}{ccc} 1 & 0 & \alpha \\ 0 & 1 & \beta \\ 0 & 0 & 1 \end{array}\right]\left[\begin{array}{ccc}a & b & 0 \\ c & d & 0 \\ 0 & 0 & 1 \end{array}\right]$
\end{inner_problem}

We multiply:

$$\begin{bmatrix} 1 & 0 & \alpha \\ 0 & 1 & \beta \\ 0 & 0 & 1 \end{bmatrix} \begin{bmatrix} a & b & 0 \\ c & d & 0 \\ 0 & 0 & 1 \end{bmatrix} = \begin{bmatrix}
a & b & \alpha \\ c & d & \beta \\ 0 & 0 & 1 \end{bmatrix}.$$

Well that was easy. In the same logic as the previous subproblem, this multiplication represents a 2D matrix transformation by $\left[\begin{smallmatrix} a & b \\ c & d \end{smallmatrix}\right]$, then a translation by $\langle \alpha, \beta, 0\rangle$.

\begin{outer_problem}
\item What does each of these matrices represent?
\end{outer_problem}

\begin{inner_problem}[start=1]
\item $\left[\begin{array}{ccc} a & b & \alpha \\ c & d & \beta \\ 0 & 0 & 1 \end{array}\right]$
\end{inner_problem}

This is the result to Problem~14b, so it's a 2D matrix transformation by $\left[\begin{smallmatrix} a & b \\ c & d \end{smallmatrix}\right]$, then a translation by $\langle \alpha, \beta, 0\rangle$.

\begin{inner_problem}
\item $\left[\begin{array}{ccc} \cos\theta & -\sin\theta & \alpha \\ \sin\theta & \cos\theta & \beta \\ 0 & 0 & 1 \end{array}\right]$
\end{inner_problem}

We recognize the 2D matrix transformation here as a rotation by $\theta$ radians counterclockwise. Thus, this is a rotation by $\theta$ radians counterclockwise, followed by a translation by $\langle \alpha, \beta, 0\rangle$.

\begin{outer_problem}
\item
\end{outer_problem}

\begin{inner_problem}[start=1]
\item Rewrite your translation matrix and your preimage vector from Problem~\ref{prob:translation_matrix} so that you do not restrict your translations to the plane $z=1$, but can translate in the $x$, $y$, and $z$ directions. (Hint: think four dimensions!)
\end{inner_problem}

We do as the problem hints, and construct a $4\times 4$ matrix which translates by $\langle \alpha, \beta, \gamma, 0\rangle$ in four dimensions.

$$\begin{bmatrix} 1 & 0 & 0 & \alpha \\ 0 & 1 & 0 & \beta \\ 0 & 0 & 1 & \gamma \\ 0 & 0 & 0 & 1 \end{bmatrix} \begin{bmatrix} x \\ y \\ z \\ 1 \end{bmatrix} = \begin{bmatrix} x + \alpha \\ y + \beta \\ z + \gamma \\ 1\end{bmatrix}.$$

\begin{inner_problem}
\item Write a matrix product that translates the point $(2,3,-5)$ by the vector $(4,-1,2)$.
\end{inner_problem}

We have $(x,y,z)=(2,3,-5)$ and $(\alpha,\beta,\gamma) = (4,-1,2)$:

$$\begin{bmatrix} 1 & 0 & 0 & 4 \\ 0 & 1 & 0 & -1 \\ 0 & 0 & 1 & 2 \\ 0 & 0 & 0 & 1 \end{bmatrix} \begin{bmatrix} 2 \\ 3 \\ -5 \\ 1 \end{bmatrix} = \begin{bmatrix} 6 \\ 2 \\ -3 \\ 1\end{bmatrix}.$$

\end{document}
