% Due to the usage of the subfiles module, every subfile of the document (sans the cover file) uses this exact preamble.
\documentclass[8pt, twosided, a4paper]{article} % 8 point font, A4-sized paper

\usepackage{subfiles}
\usepackage{pdfpages} % Include PDF

% We use inline Asymptote throughout this document. We could split them out into their own .asy files, but ehhhhh
\usepackage[inline]{asymptote}

% UTF-8 encoding in files
\usepackage[utf8]{inputenc}

% Document styling information
\usepackage[margin=2.5cm]{geometry}
\usepackage[scaled]{helvet} % Helvetica
\renewcommand\familydefault{\sfdefault} % Use Helvetica for bolded text too
\usepackage[T1]{fontenc}

% Tikz stuff
\usepackage{tikz}
\usetikzlibrary{matrix, fit, tikzmark, calc}

% Other packages
\usepackage{amsfonts}
\usepackage{array}
\usepackage{blkarray}
\usepackage{caption}
\usepackage{cancel}
\usepackage{color, colortbl}
\usepackage[us, 12hr]{datetime}
\usepackage{enumitem}
\usepackage{float}
\usepackage[bottom]{footmisc}
\usepackage{hhline}
\usepackage{mathtools}
\usepackage{multicol}
\usepackage{multirow}
\usepackage{setspace}
\usepackage{wrapfig}

% Macro definitions
\definecolor{light-gray}{gray}{0.95}
\newcommand{\twomat}[4] {\left[\begin{array}{cc} #1 & #2 \\ #3 & #4 \end{array}\right]}
\newcommand{\threemat}[9] {\left[\begin{array}{ccc} #1 & #2 & #3 \\ #4 & #5 & #6 \\ #7 & #8 & #9 \end{array}\right]}

% i ended up using array instead of smallmatrix and stwovec in many cases. usually because it looked better and was easier to read, and was consistent with everything else. the only reason i could see these being used is in cases of space saving but i feel like there could be other solutions to that. sorry though if i misunderstood something and should not have done that. -b
\newcommand{\twovec}[2] {\left[\begin{array}{c} #1 \\ #2 \end{array}\right]}
\newcommand{\stwovec}[2] {\left[\begin{smallmatrix} #1 \\ #2 \end{smallmatrix}\right]}

% Counters for problems and subproblems
\newcounter{problem_i}
\newcounter{problem_ii}

% END PREAMBLE




\begin{document}
\pagenumbering{gobble}

% too dense without it
\setlength{\parskip}{1em}

\includepdf[noautoscale=true,pages=-,width=\paperwidth]{./cover/cover_source.pdf} % cover

\subfile{credits/credits_source.tex}
\setcounter{page}{0}
\pagebreak
\pagenumbering{arabic}

\restoregeometry % credits have a different margin setting

\setlength{\parskip}{0.5em}

\tableofcontents
\pagebreak

% Reset to correct value
% Heyyyyy so idk what the correct parskip really is (0 or 2.5em so sorry if i messed this up) -B
\setlength{\parskip}{0em}

\subfile{trig_review/trig_review_source.tex} % trig review
\pagebreak
\setcounter{figure}{0}
\subfile{itsasnap/itsasnap_source.tex}
\pagebreak
\setcounter{figure}{0}
\subfile{snap_flip/snap_flip_source.tex}
\pagebreak
\setcounter{figure}{0}
\subfile{rrg/rrg_source.tex} % Rotation reflection groups
\pagebreak
\setcounter{figure}{0}
\subfile{inf/inf_source.tex} % Infinite groups
\pagebreak
\setcounter{figure}{0}
\subfile{cmplx_geo/cmplx_geo_source.tex} % Geometry of complex numbers
\pagebreak
\setcounter{figure}{0}
\subfile{vitamin_i/vitamin_i_source.tex} % Vitamin i
\pagebreak
\setcounter{figure}{0}
\subfile{mtrx_mult/mtrx_mult_source.tex} % Matrix multiplication
\pagebreak
\setcounter{figure}{0}
\subfile{map_plane/map_plane_source.tex} % Mapping the Plane
\pagebreak
\setcounter{figure}{0}
\subfile{plane_rot/plane_rot_source.tex} % Plane rotate
\pagebreak
\setcounter{figure}{0}
\subfile{mat_gen/mat_gen_source.tex} % Matrices generate groups
\pagebreak
\setcounter{figure}{0}
\subfile{comp_map/comp_map_source.tex} % Composite mappings (fat one)
\pagebreak
\setcounter{figure}{0}
\subfile{inverses/inverses_source.tex} % Composite mappings (fat one)
\pagebreak
\setcounter{figure}{0}
\subfile{mod_m/mod_m_source.tex} % Modulo m meets groups
\pagebreak
\setcounter{figure}{0}
\subfile{eigen/eigen_source.tex} % eigenvectors and eigenvalues
\pagebreak
\setcounter{figure}{0}
\subfile{comp_func/comp_func_source.tex} % Composition of functions
\pagebreak
\setcounter{figure}{0}
\subfile{glossary/glossary.tex} % Glossary
\pagebreak

\end{document}
