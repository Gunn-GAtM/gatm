\documentclass[../gatm_answers.tex]{subfiles}

\begin{document}

\section{Eigenvectors and Eigenvalues}

\begin{outer_problem}[start=1]
\item Consider the matrix equation $\twomat{0}{1}{6}{1}\twovec{x}{y} = \twovec{y}{6x+y}=\twovec{x'}{y'}$. We wish to find an eigenvector $\twovec{x}{y}$.
\end{outer_problem}

\begin{inner_problem}[start=1]
\item On graph paper, draw what the matrix $\twomat{0}{1}{6}{1}$ does to the vectors $\twovec{1}{0}$ and $\twovec{0}{1}$.
\end{inner_problem}

\begin{inner_problem}
\item In your picture, draw a rough line through the origin where you think a family of eigenvectors may be.
\end{inner_problem}

\begin{inner_problem}
\item Try some lattice points, say $\stwovec{1}{1}$, $\stwovec{1}{2}$, $\stwovec{1}{3}$, $\stwovec{1}{4}$, $\stwovec{1}{5}$. What does the matrix transform each vector into?
\end{inner_problem}

\begin{inner_problem}
\item Which of these is an eigenvector?
\end{inner_problem}

\begin{inner_problem}
\item Does it lie near the line you drew earlier?
\end{inner_problem}

\begin{outer_problem}
\item This guess-and-check process for finding eigenvectors is terrible, so let's develop a procedure to find the eigenvalues and eigenvectors for any $2\times 2$ matrix. We will use the same example.

\begin{align*}
\twomat{0}{1}{6}{1}\twovec{x}{y} &= \lambda\twovec{x}{y} & \text{Definition of eigenvector} \\
&= \lambda\twomat{1}{0}{0}{1}\twovec{x}{y} \\
\Longrightarrow \left(\twomat{0}{1}{6}{1}-\lambda\twomat{1}{0}{0}{1}\right)\twovec{x}{y}&=\twovec{0}{0} & \text{Subtraction and factoring} \\
\Longrightarrow \twomat{-\lambda}{1}{6}{1-\lambda}\twovec{x}{y} &= \twovec{0}{0}
\end{align*}
\end{outer_problem}

\begin{inner_problem}[start=1]
\item If $\stwovec{x}{y}\neq \stwovec{0}{0}$, then $$\det \twomat{-\lambda}{1}{6}{1-\lambda}=0.$$ Why? Think inverses.
\end{inner_problem}

\begin{inner_problem}
\item Find the above determinant in terms of $\lambda$ and solve for the eigenvalues.
\end{inner_problem}

\begin{inner_problem}
\item One eigenvalue is $\lambda=3$. We solve for the associated eigenvector like so:
\begin{align*}
\twovec{0}{0} &= \twomat{-\lambda}{1}{6}{1-\lambda}\twovec{x}{y} \\
&= \twomat{-3}{1}{6}{-2}\twovec{x}{y} \\
\Longrightarrow \twovec{0}{0} &= \twovec{-3x+y}{6x-2y} \\
\Longrightarrow y&=3x \rightarrow \twovec{x}{y}=s\twovec{1}{3}
\end{align*}
Solve for the other eigenvector using the other eigenvalue from part (b).
\end{inner_problem}

\begin{inner_problem}
\item Check your work by multiplying the matrix by the eigenvector!
\end{inner_problem}

\begin{outer_problem}
\item Solve for the eigenvectors and eigenvalues of the following matrices:
\end{outer_problem}

\begin{inner_problem}[start=1]
\item $\twomat{3}{24}{4}{7}$
\end{inner_problem}

\begin{inner_problem}
\item $\twomat{3}{1}{2}{4}$
\end{inner_problem}

\begin{inner_problem}
\item $\twomat{1}{-1}{4}{6}$
\end{inner_problem}

\begin{outer_problem}
\item Fill in the blanks: The image of an eigenvector will stay the same \underline{\phantom{00000}} through the \underline{\phantom{00000}} when acted on by any power of the transformation \underline{\phantom{00000}} for which it is an eigenvector. Therefore, the image of the eigenvector is simply the \underline{\phantom{00000}} of the \underline{\phantom{00000}} itself by its corresponding \underline{\phantom{00000}}.
\end{outer_problem}

\begin{outer_problem}
\item
\end{outer_problem}

\begin{inner_problem}[start=1]
\item If the transformation matrix were a reflection over a line $y=x\tan\theta$, in what directions would the two eigenvectors point? Think geometrically.
\end{inner_problem}

\begin{inner_problem}
\item What would the angle between them be?
\end{inner_problem}

\begin{inner_problem}
\item What would their eigenvalues be?
\end{inner_problem}

\begin{outer_problem}
\item Recall that multiplication by $\twomat{\cos 2\theta}{\sin 2\theta}{\sin 2\theta}{-\cos 2\theta}$ results in a reflection over $y=x\tan \theta$.
\end{outer_problem}

\begin{inner_problem}[start=1]
\item Write a matrix that results in a reflection over the line $y=\frac{\sqrt{3}}{3}x.$
\end{inner_problem}

\begin{inner_problem}
\item Find the eigenvalues of that matrix, and the corresponding eigenvectors.
\end{inner_problem}

\begin{inner_problem}
\item Do your calculations agree with your answers to the previous problem?
\end{inner_problem}

\begin{inner_problem}
\item What are the relationships between the two eigenvectors and between the two eigenvalues?
\end{inner_problem}

\begin{outer_problem}
\item
\end{outer_problem}

\begin{inner_problem}[start=1]
\item Write a matrix which results in a $60^\circ$ rotation counterclockwise.
\end{inner_problem}

\begin{inner_problem}
\item Find the eigenvalues. What do you find strange?
\end{inner_problem}

\begin{inner_problem}
\item Find the eigenvectors for those eigenvalues. What's strange about them?
\end{inner_problem}

\begin{inner_problem}
\item Explain what's going on.
\end{inner_problem}

\begin{inner_problem}
\item What are the relationships between the two eigenvectors and between the two eigenvalues?
\end{inner_problem}

\begin{outer_problem}
\item The matrix $\twomat{1}{2}{0}{1}$ is a shear parallel to the $x$ axis.
\end{outer_problem}

\begin{inner_problem}[start=1]
\item What vectors don't change direction when multiplied by this matrix?
\end{inner_problem}

\begin{inner_problem}
\item What would you expect the eigenvectors to be?
\end{inner_problem}

\begin{inner_problem}
\item Find the eigenvectors and eigenvalues of this matrix.
\end{inner_problem}

\begin{inner_problem}
\item What is different this time?
\end{inner_problem}

\begin{inner_problem}
\item Can you represent every vector as sums of eigenvectors?
\end{inner_problem}

\begin{outer_problem}
\item The matrices $\left[\begin{smallmatrix} 2 & 0 \\ 0 & 5 \end{smallmatrix}\right]$ and $\left[\begin{smallmatrix} 3 & 0 \\ 0 & 3 \end{smallmatrix}\right]$ result in some stretches. Find the eigenvectors and eigenvalues for both.
\end{outer_problem}

\begin{outer_problem}
\item Note that most $2\times 2$ matrices have two eigenvectors. How many would you expect to find for an $n\times n$ matrix?
\end{outer_problem}

\begin{outer_problem}
\item Assuming that $p,q,r,s,t,u,x,y$ are real, what conditions would you impose on them in the matrices (i)~$\twomat{3}{p}{q}{4}$, (ii)~$\twomat{x}{-2}{3}{y}$, and (iii)~$\twomat{r}{s}{t}{u}$ to have...
\end{outer_problem}

\begin{inner_problem}[start=1]
\item ... two real eigenvalues?
\end{inner_problem}

\begin{inner_problem}
\item ... two complex eigenvalues?
\end{inner_problem}

\begin{inner_problem}
\item ... only one eigenvalue?
\end{inner_problem}

\begin{outer_problem}
\item
\end{outer_problem}

\begin{inner_problem}[start=1]
\item Write a $3\times 3$ matrix showing a rotation of $\theta$ around the $z$ axis.
\end{inner_problem}

\begin{inner_problem}
\item Name the real eigenvector (this shouldn't require any work).
\end{inner_problem}

\begin{inner_problem}
\item Find all three eigenvectors.
\end{inner_problem}

\begin{outer_problem}
\item
\end{outer_problem}

\begin{inner_problem}
\item What should the absolute value of an eigenvalue of any rotation matrix be?
\end{inner_problem}

\begin{inner_problem}
\item The complex eigenvalues relate to the angle of rotation. What is that relationship?
\end{inner_problem}

\begin{outer_problem}
\item In a right-handed coordinate system, rotations in three dimensions are performed by combinations of the three matrices
$$X=\threemat{1}{0}{0}{0}{\cos\alpha}{-\sin\alpha}{0}{\sin\alpha}{\cos\alpha},\, Y=\threemat{\cos\beta}{0}{\sin\beta}{0}{1}{0}{-\sin\beta}{0}{\cos\beta},\, Z=\threemat{\cos\gamma}{-\sin\gamma}{0}{\sin\gamma}{\cos\gamma}{0}{0}{0}{1}.$$
Each matrix $X,Y,Z$ rotates around the $x,y,z$ axes by $\alpha,\beta,\gamma$, respectively.

In 2D, rotations combine to make other rotations. Similarly, if we combine any number of these rotations, the net result will be a rotation about some axis---though not necessarily a \textit{coordinate} axis. Another way to picture this is that if we operate on an origin-centered sphere with these matrices, there will always be two opposite points\footnote{These are often called antipodes.} on the sphere which have no net movement.

Try computing the following products.
\end{outer_problem}

\begin{inner_problem}[start=1]
\item $XY$
\end{inner_problem}

\begin{inner_problem}
\item $XZ$
\end{inner_problem}

\begin{inner_problem}
\item $YX$
\end{inner_problem}

\begin{inner_problem}
\item $ZX$
\end{inner_problem}

\begin{outer_problem}
\item
\end{outer_problem}

\begin{inner_problem}[start=1]
\item Without matrices, consider a cube with side length $2$ at the origin so its faces are perpendicular to the coordinate axes. Rotate it, first $90^\circ$ about the $y$ axis, then $90^\circ$ about the $x$ axis. The net result should leave two vertices fixed. Which two?
\end{inner_problem}

\begin{inner_problem}
\item Write a vector for the axis of rotation. How many degrees do you think the net rotation of the cube is? Be careful; the answer is not $180^\circ$.
\end{inner_problem}

\begin{inner_problem}
\item Let's check our answers using matrices. Write a matrix product that corresponds to a rotation of $90^\circ$ about the $y$ axis, followed by $90^\circ$ about the $x$ axis.
\end{inner_problem}

\begin{inner_problem}
\item Multiply out the matrix product.
\end{inner_problem}

\begin{inner_problem}
\item Remember that the real eigenvector in a rotation gives the axis of rotation, and the complex eigenvalues give information about the net rotation. Evaluate these and check your answers for (a) and (b).
\end{inner_problem}

\begin{outer_problem}
\item Here are three rotation matrices:
$$\threemat{\frac{2}{3}}{-\frac{2}{3}}{-\frac{1}{3}}{\frac{1}{3}}{\frac{2}{3}}{-\frac{2}{3}}{\frac{2}{3}}{\frac{1}{3}}{\frac{2}{3}}
\threemat{\frac{7}{9}}{\frac{4}{9}}{\frac{4}{9}}{-\frac{4}{9}}{-\frac{1}{9}}{\frac{8}{9}}{\frac{4}{9}}{-\frac{8}{9}}{\frac{1}{9}}
\threemat{\cos\gamma}{-\sin\gamma}{0}{\cos\alpha\sin\gamma}{\cos\alpha\sin\gamma}{-\sin\alpha}{\sin\alpha\sin\gamma}{\sin\alpha\cos\gamma}{\cos\alpha}$$
\end{outer_problem}

\begin{inner_problem}[start=1]
\item What is the determinant of each matrix? (Don't work, think!)
\end{inner_problem}

\begin{inner_problem}
\item What is true of each row and each column?
\end{inner_problem}

\begin{inner_problem}
\item Find the axis of rotation associated with each matrix.
\end{inner_problem}

\begin{inner_problem}
\item Find the angle of rotation associated with each matrix.
\end{inner_problem}

\begin{inner_problem}
\item Decompose the last matrix into two rotations about coordinate axes.
\end{inner_problem}

\end{document}
