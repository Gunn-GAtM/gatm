\documentclass[../gatm_answers.tex]{subfiles}

\begin{document}

\section{It's a Snap}

\begin{asydef}
	int factorial(int n) { // Factorial function for convenience
		if (n == 0 || n == 1)
			return 1;
		return n * factorial(n - 1); // Tail recursion... why not?
	}
	
	// Explanation of mapping integer: Instead of representing the mappings as integer arrays, I represented them as plain integers. No clue why, but I think it's because I thought Asymptote lacked 2D arrays. Anyway, the integer 0 is the identity, which is {1,2,3} in a mapping (start col) => (end col). The integer 1 is A, which is {1,3,2}. The integer 2 is C, which is {2,1,3}... in other words, the integer is the index in the list of permutations for {1,2,...,c-1,c}, where c is the number of columns (cols)
	
	/* This function takes in a pair drawing offset "offset" describing the (x,y) of the top left corner of the snap element, an integer array of mapping integers which will be chained, integer number of columns, an x spacing, and a y spacing (for columns and rows respectively) */
	/* For example, to draw the element A connected with the element B at the location (1,1) with a spacing of 4 between columns and 6 between rows, you would effectively do
	
	drawSnapElements((1,1), // at (1,1)
					 {1,5}, // A, then B
					 3, // 3 columns
					 4, // 4 units between columns
					 6); // 6 units between rows.
	*/
	
	void drawSnapElements(pair offset, int[] mappings = {0}, int cols = 3, real xd = 1, real yd = 1.5) {
		int rows = mappings.length + 1;
		
		for (int x = 0; x < cols; ++x) {
			for (int y = 0; y < rows; ++y) {
				if (y < rows - 1) {
					int map = mappings[y];
					int[] needs; // Which columns (bottom) need to have a line drawn, used in the loop

					for (int c = 0; c < cols; ++c) needs.push(c); // Push all columns onto the needed list

					for (int i = 0; i < cols; ++i) { // This loop is used to convert an integer into the actual mapping, by figuring out which mapping it actually is
						int fact = factorial(cols - i - 1);

						int quotient = map # fact; // OCTOTHORPE = integer division lolololol
						int index = needs[quotient]; // index of column to be drawn
						needs.delete(quotient);

						map %= fact; // Get remainder

						draw((offset + (i * xd, -y * yd)) -- (offset + (index * xd, -(y + 1) * yd))); // Draw the elastic
					}
				}
				dot(offset + (x * xd, -y * yd)); // add the post dot
			}
		}
	}
	
	int[] standard_indices = {0,1,5,2,4,3}; // mapping integers corresponding to I through E
	string[] standard_labels = {"I", "A", "B", "C", "D", "E"}; // for convenience
\end{asydef}

\newcommand\snap{\bullet}

\begin{outer_problem}[start=1]
	\item Fill out a $6\times 6$ table like the one in Figure~\ref{fig:sbstable}, showing the results of each of the $36$ possible snaps, where $X\snap Y$ is in $X$'s row and $Y$'s column.
	      $A\snap B=E$ is done for you.
\end{outer_problem}

% elements = {"I" : [0,1,2], "A" : [0,2,1], "B": [2,1,0], "C": [1,0,2], "D": [2,0,1], "E": [1,2,0]}
% def snap(elem1, elem2):
%      return [elem1[elem2[n]] for n in range(3)]
% def name_from(elem):
%      for key in elements:
%             test_elem = elements[key]
%             if elem[0] == test_elem[0] && elem[1] == test_elem[1] && elem[2] == test_elem[2]:
%                    return key
%      return None
% elemlist = ["I","A","B","C","D","E"]
% for elem1key in elemlist:
%      print elem1key + " & " + " & ".join([name_from(snap(elements[elem1key],elements[elem2key])) for elem2key in elemlist]) + " \\\\"
%             
$$\begin{array}{ccccccc}
\snap & I & A & B & C & D & E \\
I & I & A & B & C & D & E \\
A & A & I & E & D & C & B \\
B & B & D & I & E & A & C \\
C & C & E & D & I & B & A \\
D & D & B & C & A & E & I \\
E & E & C & A & B & I & D \\
\end{array}$$

\begin{outer_problem}
	\item Would this table look different if you wrote the elements $A$ through $E$ in a different order?
\end{outer_problem}

Yes; here's an example:

% elemlist = ["I","E","A","D","B","C"]

$$\begin{array}{ccccccc}
\snap & I & E & A & D & B & C \\
I & I & E & A & D & B & C \\
E & E & D & C & I & A & B \\
A & A & B & I & C & E & D \\
D & D & I & B & E & C & A \\
B & B & C & D & A & I & E \\
C & C & A & E & B & D & I \\
\end{array}$$

\begin{outer_problem}
	\item Which of the elements is the \textbf{identity element} $K$, such that $X\snap K = K\snap X = X$ for all $X$? \label{prob:group_definition_start}
\end{outer_problem}

The identity element is $I$, since $I\snap A = A\snap I = A$, $I\snap B = B\snap I = B$, and so forth.

\begin{outer_problem}
	\item Does every element have an inverse; can you get to the identity element from every element using only one snap?
\end{outer_problem}

Yes you can. The inverses are shown below.

\begin{align*}
I &\leftrightarrow I \\
A &\leftrightarrow A \\
B &\leftrightarrow B \\
C &\leftrightarrow C \\
D &\leftrightarrow E \\
\end{align*}

Note that the inverse of an element $X$ is denoted $X^{-1}$.

\begin{outer_problem}
	\item
\end{outer_problem}

\begin{inner_problem}[start=1]
	\item Is the snap operation commutative (does $X \snap Y = Y \snap X$ for all $X,Y$)?
\end{inner_problem}

No, the snap operation is not commutative. For example, $A\snap B=E$, but $B\snap A = D$.

\begin{inner_problem}
	\item Is the snap operation associative (does $(X \snap Y) \snap Z = X \snap (Y\snap Z)$ for all $X,Y,Z$)?'
\end{inner_problem}

Yes, the snap operation is associative. You can rationalize this as the fact that a $4\times 3$ grid of posts is snapped to a single configuration, regardless of which middle row you remove first. This is shown in Figure~\ref{fig:four_by_three}.

\begin{figure}
\centering
\begin{minipage}{0.3\textwidth}
\begin{asy}[width=0.7\textwidth]
int[] draw_elems = {1,5,4};
drawSnapElements((0,0), draw_elems, 3);

label("$\stackrel{\text{snap!}}{\Longrightarrow}$", (3, -2.25));

int[] draw_elems2 = {3};

drawSnapElements((4,0), draw_elems2, 3, 1, 3 * 1.5);
\end{asy}
\caption{A $4\times 3$ grid of posts has a unique result after the snap operation.}
\label{fig:four_by_three}
\end{minipage}
\end{figure}

\begin{outer_problem}
	\item
\end{outer_problem}

\begin{inner_problem}[start=1]
	\item For any elements $X, Y$, is there always an element $Z$ so that $X\snap Z=Y$?
\end{inner_problem}

Yes, there is always a way to get from one element to another in one snap. You can prove this by construction. If element $X$ connects $n_1$ to $n'_1$, $n_2$ to $n'_2$, and $n_3$ to $n'_3$, and element $Y$ connects $m_1$ to $m'_1$, $m_2$ to $m'_2$, and $m_3$ to $m'_3$, then the solution $Z$ to $X\snap Z=Y$ connects $m_1$ to $n_{m'_1}$, $m_2$ to $n_{m'_2}$, and $m_3$ to $n_{m'_3}$.

That's probably a bit hard to understand, but a more clever solution uses inverses. We multiply $X$ by $X^{-1}$, then by $Y$: $$X\snap X^{-1}\snap Y = Y.$$ But since every element has an inverse, and the snap operation is associative, we have

\begin{align*}
X\snap (X^{-1} \snap Y) &= Y \\
\Longrightarrow Z = X^{-1} \snap Y.
\end{align*}

In this way, we have constructed the element $Z$.

\begin{inner_problem}
	\item For (a), is $Z$ always unique?
\end{inner_problem}

Yes. To show this, we use a proof by contradiction. Suppose we have two solutions $Z_1$ and $Z_2$ so that $Z_1\neq Z_2$ and

\begin{align*}
X\snap Z_1 &= Y \\
X\snap Z_2 &= Y.
\end{align*}

We multiply to the left by $Y^{-1}$. Note that since the snap operation is not commutative, we need to multiply both sides on a specific side:

\begin{align*}
Y^{-1} \snap X\snap Z_1 &= Y^{-1} \snap Y = I \\
Y^{-1} \snap X\snap Z_2 &= I
\end{align*}

So $Z_1,Z_2$ are the inverses of $Y^{-1}\snap X$. But the inverse of an element is unique; we've showed this by listing them all out! Thus, $Z_1=Z_2$, contradicting our assumption and proving that $Z$ is unique in $X\snap Z = Y$.

\begin{outer_problem}
	\item If you constructed a $5\times 5$ table using only $5$ of the snap elements, the table would not describe a group, because there would be entries in the table not in those $5$.
	      Therefore, a group must be \textbf{closed} under its operation; if $X,Y\in G$ ($\in$ means ``is/are in''), then $X\snap Y\in G$ for all $X,Y$.
	      Some subsets, however, do happen to be closed.

	      Write valid group tables using exactly $1$, $2$, and $3$ elements from the snap group. These are known as \textbf{subgroups}.\label{prob:group_definition_end}
\end{outer_problem}

\begin{outer_problem}
	\item What do you guess is the complete definition of a mathematical group?
	      (Hint: consider your answers to Problems \ref{prob:group_definition_start}--\ref{prob:group_definition_end}.)
	\item Notice that $E\snap E\snap E=I$. (See Figure~\ref{fig:eper3}.)
	      This means that $E$ has a \textbf{period} of $3$ when acting upon itself.
	      Which elements have a period of
	      \begin{multicols}{3}
		      \begin{enumerate}
			      \item $1$?
			      \item $2$?
			      \item $3$?
		      \end{enumerate}
	      \end{multicols}
	\item Answer the following with the $1$, $2$, and $4$-post snap groups $S_1$, $S_2$ and $S_4$.\begin{enumerate}
		      \item How many elements would there be?
		      \item Systematically draw and name them.
		      \item Make a group table of these elements.
		            For $4$ posts, instead of creating the massive table, give the number of entries that table would have.
		      \item What is the relationship between this new table and your original table?
	      \end{enumerate}
	\item Can you think of an easier way to generate a snap group table without drawing all the possible configurations?
	\item \begin{enumerate}
		      \item How many elements would there be in the $5$-post snap group? \label{prob:five_post_snap_list_start}
		      \item How many entries would its table have?
		      \item What possible periods would its elements have? \label{prob:five_post_snap_list_end}
		      \item Extend your answers for \ref{prob:five_post_snap_list_start}--\ref{prob:five_post_snap_list_end} to $M$ posts per row.
	      \end{enumerate}
	\item As we learned, a \textbf{permutation} of some things is an order in which they can be arranged.
	      What is the relationship between the set of permutations of $m$ things and the $m$-post snap group?
\end{outer_problem}

\end{document}