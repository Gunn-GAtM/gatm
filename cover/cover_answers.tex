\documentclass[11pt, a4paper]{article}
% 11 point font, A4-sized paper

\usepackage[utf8]{inputenc}
% Specifies that the input is UTF-8

\usepackage[margin=.2in]{geometry}

\usepackage[inline]{asymptote}

\usepackage{anyfontsize}

\usepackage[scaled]{helvet} % Helvetica again, for consistency
\renewcommand\familydefault{\sfdefault}
\usepackage[T1]{fontenc}

\renewcommand{\baselinestretch}{1.2}

\begin{document}

\begin{titlepage}
	% Title page
	\vspace*{\fill}

    \begin{center}

        % Cover page figure
		
		\begin{flushleft}
			\fontsize{32}{32}\selectfont{\textbf{A Geometric Approach\\To Matrices}}\\
			\vspace{0.5in}
			\fontsize{24}{24}\selectfont{\textbf{Answer Key}}\\
			\vspace{0.2in}
			\fontsize{16}{16}\selectfont{\textbf{Timothy Herchen}\\Henry M. Gunn High School\\Analysis Honors}
		\end{flushleft}
	
		\vspace*{-2.5in}
		\hspace*{4.25in}
       	\begin{asy}
			size(480); // Sets the size of the cover page figure

			pair cis(real theta) { // Gives points on the unit circle
				return (cos(theta), sin(theta));
			}

       		pair A = cis(-pi/6), B = cis(pi/2), C = cis(7*pi/6); // Starting triangle corners

		    for (int i = 0; i < 4; ++i) { // Draws the triangles and dots
			    draw(A--B--C--cycle); // the triangle

			    dot(A); // the dots
			    dot(B);
			    dot(C);

		    		if (i < 3) {
					A *= -2; B *= -2; C *= -2; // Scales A, B, C through (0,0) to the next triangle
				}
		    }

		    dot((0,0)); // Center dot

		    draw(A--(-1/2 * A), dashed); // draw the transformation between the current triangle and previous
		    draw(B--(-1/2 * B), dashed);
		    draw(C--(-1/2 * C), dashed);
		    
		    draw(circle((0,0), 8)); // draw the unit circle surrounding the whole thing
       	\end{asy}
       	
    \end{center}

	\vspace*{\fill} % along with top copy of this: centers the page
\end{titlepage}


\end{document}
